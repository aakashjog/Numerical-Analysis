\documentclass[fleqn, a4paper, 11pt, oneside]{amsart}
%\usepackage[top = 2cm, bottom = 1cm, left = 1cm, right = 1cm]{geometry}
\usepackage{exsheets, tasks}
\usepackage{amsmath, amssymb, amsthm} %standard AMS packages
\usepackage{marginnote} %marginnotes
\usepackage{gensymb} %miscellaneous symbols
\usepackage{commath} %differential symbols
\usepackage{xcolor} %colours
\usepackage{cancel} %cancelling terms
\usepackage[free-standing-units, space-before-unit]{siunitx} %formatting units
\usepackage{tikz, pgfplots} %diagrams
\usetikzlibrary{calc, hobby, patterns, intersections, decorations.markings}
\usepackage{graphicx} %inserting graphics
\usepackage{hyperref} %hyperlinks
\usepackage{datetime} %date and time
\usepackage{ulem} %underline for \emph{}
\usepackage{xfrac} %inline fractions
\usepackage{enumerate,enumitem} %numbered lists
\usepackage{float} %inserting floats
\usepackage{circuitikz}[american voltages, american currents] %circuit diagrams

\newcommand\numberthis{\addtocounter{equation}{1}\tag{\theequation}} %adds numbers to specific equations in non-numbered list of equations

\newcommand{\AxisRotator}[1][rotate=0]{
	\tikz [x=0.25cm,y=0.60cm,line width=.2ex,-stealth,#1] \draw (0,0) arc (-150:150:1 and 1);%
} %rotation symbols on axes

\theoremstyle{definition}
\newtheorem{example}{Example}
\newtheorem{definition}{Definition}

\theoremstyle{theorem}
\newtheorem{theorem}{Theorem}

\newcommand{\curl}{\mathrm{curl\,}}

\makeatletter
\@addtoreset{section}{part} %resets section numbers in new part
\makeatother

\renewcommand{\thesubsection}{(\arabic{subsection})}
\renewcommand{\thesection}{(\arabic{section})}

%section headings on left
\makeatletter
\def\specialsection{\@startsection{section}{1}%
	\z@{\linespacing\@plus\linespacing}{.5\linespacing}%
	%  {\normalfont\centering}}% DELETED
	{\normalfont}}% NEW
\def\section{\@startsection{section}{1}%
	\z@{.7\linespacing\@plus\linespacing}{.5\linespacing}%
	%  {\normalfont\scshape\centering}}% DELETED
	{\normalfont\scshape}}% NEW
\makeatother

%forces newline after subsection
\makeatletter
\def\subsection{\@startsection{subsection}{3}%
	\z@{.5\linespacing\@plus.7\linespacing}{.1\linespacing}%
	{\normalfont\itshape}}
\makeatother

\settasks{counter-format = tsk[1].}

\SetupExSheets{solution/print = true}

%opening
\title{Numerical Analysis : Assignment 7}
\author
{
	Aakash Jog\\
	ID : 989323563
}
\date{\formatdate{1}{12}{2015}}

\begin{document}

\tikzset{->-/.style={decoration={
  markings,
  mark=at position #1 with {\arrow{>}}},postaction={decorate}}}

\maketitle
%\setlength{\mathindent}{0pt}

\begin{question}
	Let
	\begin{align*}
		A &=
			\begin{pmatrix}
				1 & 1 & 1  \\
				2 & 4 & 8  \\
				3 & 9 & 27 \\
			\end{pmatrix}
	\end{align*}
	\begin{enumerate}
		\item
			Find the LU decomposition of $A$.
		\item
			Solve
			\begin{align*}
				A x &= b\\
				&=
					\begin{pmatrix}
						7  \\
						28 \\
						69 \\
					\end{pmatrix}
			\end{align*}
			using
			\begin{align*}
				A x & = L U x \\
                                    & = b
			\end{align*}
	\end{enumerate}
\end{question}

\begin{solution}
	\begin{enumerate}[leftmargin=*]
		\item
			\begin{align*}
				&
					\begin{pmatrix}
						1 & 1 & 1  \\
						2 & 4 & 8  \\
						3 & 9 & 27 \\
					\end{pmatrix}\\
				\xrightarrow{R_2 \to R_2 - 2 R_1}&
					\begin{pmatrix}
						1 & 1 & 1  \\
						0 & 2 & 6  \\
						3 & 9 & 27 \\
					\end{pmatrix}\\
				\xrightarrow{R_3 \to R_3 - 3 R_1}&
					\begin{pmatrix}
						1 & 1 & 1  \\
						0 & 2 & 6  \\
						0 & 6 & 24 \\
					\end{pmatrix}\\
				\xrightarrow{R_3 \to R_3 - 3 R_2}&
					\begin{pmatrix}
						1 & 1 & 1 \\
						0 & 2 & 6 \\
						0 & 0 & 6 \\
					\end{pmatrix}\\
			\end{align*}
			Therefore,
			\begin{align*}
				U &=
					\begin{pmatrix}
						1 & 1 & 1 \\
						0 & 2 & 6 \\
						0 & 0 & 6 \\
					\end{pmatrix}\\
				L &=
					\begin{pmatrix}
						1 & 0 & 0 \\
						2 & 1 & 0 \\
						3 & 3 & 1 \\
					\end{pmatrix}
			\end{align*}
		\item
			\begin{align*}
				A x              & = b \\
				\therefore L U x & = b
			\end{align*}
			Let
			\begin{align*}
				y & = U x
			\end{align*}
			Therefore,
			\begin{align*}
				L y &= b\\
				\therefore
					\begin{pmatrix}
						1 & 0 & 0 \\
						2 & 1 & 0 \\
						3 & 3 & 1 \\
					\end{pmatrix}
					\begin{pmatrix}
						y_1 \\
						y_2 \\
						y_3 \\
					\end{pmatrix}
				&=
					\begin{pmatrix}
						7  \\
						28 \\
						69 \\
					\end{pmatrix}
			\end{align*}
			Therefore
			\begin{align*}
				y_1                 & = 7  \\
				2 y_1 + y_2         & = 28 \\
				3 y_1 + 3 y_2 + y_3 & = 69
			\end{align*}
			Therefore, solving,
			\begin{align*}
				y_1 & = 7  \\
				y_2 & = 14 \\
				y_3 & = 6
			\end{align*}
			Therefore,
			\begin{align*}
				y &= U x\\
				\therefore
					\begin{pmatrix}
						7  \\
						14 \\
						6  \\
					\end{pmatrix}
				&=
					\begin{pmatrix}
						1 & 1 & 1 \\
						0 & 2 & 6 \\
						0 & 0 & 6 \\
					\end{pmatrix}
					\begin{pmatrix}
						x_1 \\
						x_2 \\
						x_3 \\
					\end{pmatrix}
			\end{align*}
			Therefore,
			\begin{align*}
				x_1 + x_2 + x_3 & = 7  \\
				2 x_2 + 6 x_3   & = 14 \\
				6 x_3           & = 6
			\end{align*}
			Therefore, solving
			\begin{align*}
				x_1 & = 2 \\
				x_2 & = 4 \\
				x_3 & = 1
			\end{align*}
			Therefore,
			\begin{align*}
				x &=
					\begin{pmatrix}
						2 \\
						4 \\
						1 \\
					\end{pmatrix}
			\end{align*}
	\end{enumerate}
\end{solution}

\begin{question}
	In this exercise we demonstrate the importance of row permutations even in the case
	\begin{align*}
		{a_{m,m}}^{(m)} &\neq 0
	\end{align*}
	where ${a_{m,m}}^{(m)}$ is the $m$th entry in the diagonal at step $m$.
	We show that row pivoting is important when $a_{m,m}^{(m)}$ is small.\\
	Consider the system
	\begin{align*}
		A x &=
			\begin{pmatrix}
				10^{-5} & 1 \\
				1       & 1 \\
			\end{pmatrix}
			\begin{pmatrix}
				x_1 \\
				x_2 \\
			\end{pmatrix}\\
		&=
			\begin{pmatrix}
				1 \\
				0 \\
			\end{pmatrix}
	\end{align*}
	We want to solve the system using floating point arithmetic in base 10, and mantissa with 4 digits.\\
	Note that the accurate solution to this system is
	\begin{align*}
		x_1 & = \frac{-1}{1 - 10^{-5}} \\
                    & \approx -1               \\
		x_2 & = \frac{1}{1 - 10^{-5}}  \\
                    & \approx 1
	\end{align*}
	\begin{enumerate}
		\item
			Solve the system using Gauss elimination without pivoting, in floating point arithmetic.
			Namely, at each step, only use 4 digits.
			Is the solution a good approximation to the accurate solution?
		\item
			Solve the system using Gauss elimination with pivoting, in floating point arithmetic.
			Namely, at each step, only use 4 digits.
			Is the solution a good approximation to the accurate solution?
	\end{enumerate}
\end{question}

\begin{solution}
	\begin{enumerate}[leftmargin=*]
		\item
			\begin{align*}
				&
					\begin{pmatrix}
						0.000 & 1.000 \\
						1.000 & 1.000 \\
					\end{pmatrix}\\
				\xrightarrow{R_1 \to R_1 - R_2}&
					\begin{pmatrix}
						-1.000 & 0.000 \\
						1.000  & 1.000 \\
					\end{pmatrix}\\
				\xrightarrow{R_2 \to R_2 + R_1}&
					\begin{pmatrix}
						-1.000 & 0.000 \\
						0.000  & 1.000 \\
					\end{pmatrix}\\
			\end{align*}
			Therefore,
			\begin{align*}
				U &=
					\begin{pmatrix}
						-1.000 & 0.000 \\
						0.000  & 1.000 \\
					\end{pmatrix}\\
				L &=
					\begin{pmatrix}
						1.000 & 0.000 \\
						0.000 & 1.000 \\
					\end{pmatrix}
			\end{align*}
			Therefore
			\begin{align*}
				A x              & = b \\
				\therefore L U x & = b
			\end{align*}
			Let
			\begin{align*}
				y & = U x
			\end{align*}
			Therefore,
			\begin{align*}
				L y &= B\\
				\therefore
					\begin{pmatrix}
						1 & 0 \\
						0 & 1 \\
					\end{pmatrix}
					\begin{pmatrix}
						y_1 \\
						y_2 \\
					\end{pmatrix}
				&=
					\begin{pmatrix}
						1 \\
						0 \\
					\end{pmatrix}
			\end{align*}
			Therefore,
			\begin{align*}
				y_1 & = 1 \\
				y_2 & = 0
			\end{align*}
			Therefore,
			\begin{align*}
				y &= U x\\
				\therefore
					\begin{pmatrix}
						1 \\
						0 \\
					\end{pmatrix}
				&=
					\begin{pmatrix}
						-1 & 0 \\
						0  & 1 \\
					\end{pmatrix}
					\begin{pmatrix}
						x_1 \\
						x_2 \\
					\end{pmatrix}
			\end{align*}
			Therefore,
			\begin{align*}
				-x_1 & = 1 \\
				x_2  & = 0
			\end{align*}
			Therefore, solving,
			\begin{align*}
				x_1 & = -1 \\
				x_2 & = 0
			\end{align*}
			Therefore, this solution is not a good approximation to the accurate solution.
		\item
			Let
			\begin{align*}
				V &=
					\begin{pmatrix}
						2 \\
						1 \\
					\end{pmatrix}
			\end{align*}
			Therefore,
			\begin{align*}
				&
					\begin{pmatrix}
						1.000 & 1.000 \\
						0.000 & 1.000 \\
					\end{pmatrix}\\
				\xrightarrow{} &
					\begin{pmatrix}
						1.000 & 1.000 \\
						0.000 & 1.000 \\
					\end{pmatrix}\\
			\end{align*}
			Therefore,
			\begin{align*}
				U &=
					\begin{pmatrix}
						1.000 & 1.000 \\
						0.000 & 1.000 \\
					\end{pmatrix}\\
				L &=
					\begin{pmatrix}
						1.000 & 0.000 \\
						0.000 & 1.000 \\
					\end{pmatrix}
			\end{align*}
			Using $V$,
			\begin{align*}
				B &\to
					\begin{pmatrix}
						0 \\
						1 \\
					\end{pmatrix}
			\end{align*}
			Let
			\begin{align*}
				y & = U x
			\end{align*}
			Therefore,
			\begin{align*}
				L y &= B\\
				\therefore
					\begin{pmatrix}
						1 & 0 \\
						0 & 1 \\
					\end{pmatrix}
					\begin{pmatrix}
						y_1 \\
						y_2 \\
					\end{pmatrix}
				&=
					\begin{pmatrix}
						0 \\
						1 \\
					\end{pmatrix}
			\end{align*}
			Therefore,
			\begin{align*}
				y_1 & = 0 \\
				y_2 & = 1
			\end{align*}
			Therefore,
			\begin{align*}
				y &= U x\\
				\therefore
					\begin{pmatrix}
						0 \\
						1 \\
					\end{pmatrix}
				&=
					\begin{pmatrix}
						1 & 1 \\
						0 & 1 \\
					\end{pmatrix}
					\begin{pmatrix}
						x_1 \\
						x_2 \\
					\end{pmatrix}
			\end{align*}
			Therefore,
			\begin{align*}
				x_1 + x_2 & = 0 \\
				x_2       & = 1
			\end{align*}
			Therefore, solving,
			\begin{align*}
				x_1 & = -1 \\
				x_2 & = 1
			\end{align*}
			Therefore, this solution is a good approximation to the accurate solution.
	\end{enumerate}
\end{solution}

\begin{question}
	For $v \in \mathbb{R}^n$, define
	\begin{align*}
		\|v\|_{\infty} & = \max\limits_{1 \le k \le n} |v_k| \\
		{\|v\|_1}'     & = \frac{1}{n} \sum\limits_{k = 1}^{n} |v_k|
	\end{align*}
	A different dose of medicine is given to a patient at days $1$, $2$, \dots, $100$.
	There is a theoretical ``dose vector'' $v \in \mathbb{R}^{100}$ that holds the optimal dose $v_k$, to give each day $k$, in order to best treat the patient.
	In any day $k$, a dose that is too far from $v_k$ kills the patient.
	Let $\tilde{v}$ be a calculated approximation of $v$.
	We demand
	\begin{align*}
		\|e\| & = \left\| v - \tilde{v} \right\| \\
                      & < \varepsilon
	\end{align*}
	for some small $\varepsilon$.
	What norm should we use?
	Explain your answer.
\end{question}

\begin{solution}
	If
	\begin{align*}
		{\|v\|_1}' & = \frac{1}{n} \sum\limits_{k = 1}^{n} |v_k|
	\end{align*}
	is used, the norm is the average of the dosages.
	Hence, the error is smaller than the infinity norm.
	Therefore, this norm should be used.
\end{solution}

\end{document}
