\documentclass[fleqn, a4paper, 11pt, oneside]{amsart}
%\usepackage[top = 2cm, bottom = 1cm, left = 1cm, right = 1cm]{geometry}
\usepackage{exsheets, tasks}
\usepackage{amsmath, amssymb, amsthm} %standard AMS packages
\usepackage{marginnote} %marginnotes
\usepackage{gensymb} %miscellaneous symbols
\usepackage{commath} %differential symbols
\usepackage{xcolor} %colours
\usepackage{cancel} %cancelling terms
\usepackage[free-standing-units, space-before-unit]{siunitx} %formatting units
\usepackage{tikz, pgfplots} %diagrams
\usetikzlibrary{calc, hobby, patterns, intersections, decorations.markings}
\usepackage{graphicx} %inserting graphics
\usepackage{hyperref} %hyperlinks
\usepackage{datetime} %date and time
\usepackage{ulem} %underline for \emph{}
\usepackage{xfrac} %inline fractions
\usepackage{enumerate,enumitem} %numbered lists
\usepackage{float} %inserting floats
\usepackage{circuitikz}[american voltages, american currents] %circuit diagrams

\newcommand\numberthis{\addtocounter{equation}{1}\tag{\theequation}} %adds numbers to specific equations in non-numbered list of equations

\newcommand{\AxisRotator}[1][rotate=0]{
	\tikz [x=0.25cm,y=0.60cm,line width=.2ex,-stealth,#1] \draw (0,0) arc (-150:150:1 and 1);%
} %rotation symbols on axes

\theoremstyle{definition}
\newtheorem{example}{Example}
\newtheorem{definition}{Definition}

\theoremstyle{theorem}
\newtheorem{theorem}{Theorem}

\newcommand{\curl}{\mathrm{curl\,}}

\makeatletter
\@addtoreset{section}{part} %resets section numbers in new part
\makeatother

\renewcommand{\thesubsection}{(\arabic{subsection})}
\renewcommand{\thesection}{(\arabic{section})}

%section headings on left
\makeatletter
\def\specialsection{\@startsection{section}{1}%
	\z@{\linespacing\@plus\linespacing}{.5\linespacing}%
	%  {\normalfont\centering}}% DELETED
	{\normalfont}}% NEW
\def\section{\@startsection{section}{1}%
	\z@{.7\linespacing\@plus\linespacing}{.5\linespacing}%
	%  {\normalfont\scshape\centering}}% DELETED
	{\normalfont\scshape}}% NEW
\makeatother

%forces newline after subsection
\makeatletter
\def\subsection{\@startsection{subsection}{3}%
	\z@{.5\linespacing\@plus.7\linespacing}{.1\linespacing}%
	{\normalfont\itshape}}
\makeatother

\settasks{counter-format = tsk[1].}

\SetupExSheets{solution/print = true}

%opening
\title{Numerical Analysis : Assignment 1}
\author
{
	Aakash Jog\\
	ID : 989323563
}
\date{\formatdate{20}{10}{2015}}

\begin{document}

\tikzset{->-/.style={decoration={
  markings,
  mark=at position #1 with {\arrow{>}}},postaction={decorate}}}

\maketitle
%\setlength{\mathindent}{0pt}

%\begin{question}
%	Let $n$ and $m$ be positive integers and consider the binary floating point system where our machine store numbers in the form $\pm 1 . a_1 a_2 \dots a_n 2^e$, where $e \in \{0,-1,-2,\dots,-m\}$ and $a_k \in \{0,1\}$.
%	In terms of $n$ and $m$, how many numbers are there in the system?
%\end{question}
%
%\begin{solution}
%	As each $a_k \in \{0,1\}$, the total number of possible mantissas is $2^n$.\\
%	As $e \in \{0,-1,\dots,-.\}$, the total number of possible exponents is $m + 1$.\\
%	Therefore, the number of numbers, in absolute value, which can be expressed is 
%	\begin{align*}
%		p & = (2^n) (m + 1)
%	\end{align*}
%	As the sign is represented by one bit, the total number of numbers is $2 p$.\\
%	Therefore, the total number of numbers which can be represented is
%	\begin{align*}
%		2 p & = (2) (2^n) (m + 1)
%	\end{align*}
%\end{solution}

\setcounter{question}{1}

\begin{question}
	Reformulate the following functions, if needed, to avoid loss of significant digits.
	\begin{enumerate}
		\item $f(x) = \sqrt{1 + x} - \sqrt{x}$ for $x >> 1$.
		\item $f(x) = \sqrt{1 + x} - \sqrt{x}$ for $x \approx 0$.
		\item $f(x) = \sqrt{1 + x} - 1$ for $x >> 1$.
		\item $f(x) = \sqrt{1 + x} - 1$ for $x \approx 0$.
		\item $f(x) = \ln(x + 1) - \ln(x)$ for $x >> 1$.
		\item $f(x) = \ln(x + 1) - \ln(x)$ for $x \approx 0$.
	\end{enumerate}
\end{question}

\begin{solution}
	\begin{enumerate}[leftmargin=*]
		\item
			As $x >> 1$,
			\begin{align*}
				f(x) & = \sqrt{1 + x} - \sqrt{x} \\
                                     & = \sqrt{x} - \sqrt{x}     \\
                                     & = 0                       \\
                                     & = f(x + 1)
			\end{align*}
			Therefore, there is a loss of significant digits.\\
			Therefore, the function needs to be reformulated.
			\begin{align*}
				f(x) & = \sqrt{1 + x} - \sqrt{x} \\
                                     & = \frac{1}{\sqrt{1 + x} + \sqrt{x}}
			\end{align*}
			This expression can never be zero, hence, there is no loss of significant digits.
		\item
			As $x \approx 0$,
			\begin{align*}
				f(x) & = \sqrt{1 + x} - \sqrt{x} \\
                                     & = \sqrt{1}                \\
                                     & = 1                       \\
                                     & = f(x + 1)
			\end{align*}
			Therefore, there is a loss of significant digits.\\
			Therefore, the function needs to be reformulated.
			\begin{align*}
				f(x) & = \left( 1 + \frac{x}{2} - \frac{x^2}{8} + \frac{x^3}{16} - \dots \right) - \sqrt{x}
			\end{align*}
			As this expression is unique for every $x$, there is no loss of significant digits.
		\item
			As $x >> 1$,
			\begin{align*}
				f(x)            & = \sqrt{1 + x} - 1 \\
                                                & = \sqrt{x} - 1     \\
				f(x + 1)        & = \sqrt{2 + x} - 1 \\
                                                & = \sqrt{x} - 1     \\
				\therefore f(x) & = f(x + 1)
			\end{align*}
			Therefore, there is a loss of significant digits.\\
			Therefore, the function needs to be reformulated.
			\begin{align*}
				f(x)                & = \sqrt{1 + x} - 1               \\
                                                    & = \frac{x}{\sqrt{1 + x} + 1}     \\
				\therefore f(x + 1) & = \frac{x + 1}{\sqrt{2 + x} + 1} \\
                                                    & \neq f(x)
			\end{align*}
			Hence, in this formulation, there is no loss of significant digits.
		\item
			As $x \approx 0$,
			\begin{align*}
				f(x) & = \sqrt{1 + x} - 1 \\
                                     & = 0
			\end{align*}
			Therefore, there is a loss of significant digits.\\
			Therefore, the function needs to be reformulated.
			\begin{align*}
				f(x) & = \sqrt{1 + x} - 1                                                            \\
                                     & = \left( 1 + \frac{x}{2} - \frac{x^2}{8} + \frac{x^3}{16} - \dots \right) + 1 \\
                                     & = \frac{x}{2} - \frac{x^2}{8} + \frac{x^3}{16} - \dots
			\end{align*}
			Hence, in this formulation, there is no loss of significant digits.
		\item
			As $x >> 1$,
			\begin{align*}
				f(x) & = \ln(x + 1) - \ln(x)               \\
                                     & = \ln\left( \frac{x + 1}{x} \right) \\
                                     & = \ln(1)                            \\
                                     & = f(x + 1)
			\end{align*}
			Therefore, there is a loss of significant digits.\\
			Therefore, the function needs to be reformulated.
			\begin{align*}
				f(x) & = \ln\left( \frac{x + 1}{x} \right) \\
                                     & = \ln\left( 1 + \frac{1}{x} \right)
			\end{align*}
			Hence, in this formulation, there is no loss of significant digits.
		\item
			As $x \approx 0$,
			\begin{align*}
				f(x) & = \ln(x + 1) - \ln(x)               \\
                                     & = \ln\left( \frac{x + 1}{x} \right) \\
                                     & = \ln(1)                            \\
                                     & = f(x + 1)
			\end{align*}
			Therefore, there is a loss of significant digits.\\
			Therefore, the function needs to be reformulated.
			\begin{align*}
				f(x) & = \left( x - \frac{x^2}{2} + \frac{x^3}{3} - \dots \right) - \ln(x)
			\end{align*}
			Hence, in this formulation, there is no loss of significant digits.
	\end{enumerate}
\end{solution}

\begin{question}
	Let $\tilde{x}$ and $\tilde{y}$ be approximations of the numbers $x$ and $y$, respectively.\\
	Show
	\begin{enumerate}
		\item
			\begin{align*}
				\left| e_{\frac{x}{y}} \right| & \le \frac{|x| |e_y| + |y| |e_x|}{|y| |\tilde{y}|}
			\end{align*}
%		\item
%			\begin{align*}
%				\left| e_{\frac{x}{y}} \right| & \le \frac{|x| |e_y| + |y| |e_x|}{|y|^2 - |y| |e_y|}
%			\end{align*}
%		\item
%			\begin{align*}
%				\left| e_{\frac{x}{y}} \right| & \lesssim \frac{|x| |e_y| + |y| |e_x|}{|y|^2} \left( 1 + |\delta_y| \right)
%			\end{align*}
%		\item
%			\begin{align*}
%				\left| e_{\frac{x}{y}} \right| & \lesssim \left| \frac{x}{y} \right| \left( |\delta_x| + |\delta_y| \right) \left( 1 + |\delta_y| \right)
%			\end{align*}
%		\item
%			\begin{align*}
%				\left| e_{\frac{x}{y}} \right| & \lesssim \left| \frac{x}{y} \right| \left( |\delta_x| + |\delta_y| \right)
%			\end{align*}
	\end{enumerate}
\end{question}

\begin{solution}
	\begin{enumerate}[leftmargin=*]
		\item
			\begin{align*}
				\left| e_{\frac{x}{y}} \right| & = \left| \frac{x}{y} - \frac{\tilde{x}}{\tilde{y}} \right|     \\
                                                               & = \left| \frac{x \tilde{y} - \tilde{x} y}{y \tilde{y}} \right| \\
                                                               & = \left| \frac{x (y - e_y) - (x - e_x) y}{y \tilde{y}} \right| \\
                                                               & = \left| \frac{x y - x e_y - x y + y e_x}{y \tilde{y}} \right| \\
                                                               & = \left| \frac{x e_y - y e_x}{y \tilde{y}} \right|             \\
                                                               & = \frac{|x e_y - y e_x|}{|y \tilde{y}|}                        \\
                                                               & \le \frac{|x| |e_y| + |y| |e_x|}{|y| |\tilde{y}|}
			\end{align*}
			\qed
%		\item
%			\begin{align*}
%				\left| e_{\frac{x}{y}} \right| & = \frac{|x e_y - y e_x|}{|y \tilde{y}|}             \\
%                                                               & = \frac{|x e_y - y e_x|}{|y (y - e_y)}              \\
%                                                               & = \frac{|x e_y - y e_x|}{|y^2 - y e_y|}             \\
%                                                               & \le \frac{|x| |e_y| + |y| |e_x|}{|y^2 - y e_y|}     \\
%                                                               & \le \frac{|x| |e_y| + |y| |e_x|}{|y^2| - |y e_y|}   \\
%                                                               & \le \frac{|x| |e_y| + |y| |e_x|}{|y|^2 - |y| |e_y|} \\
%			\end{align*}
	\end{enumerate}
\end{solution}

%\begin{question}
%	Let $\tilde{x}$ and $\tilde{y}$ be approximations of the numbers $x$ and $y$, respectively.
%	\begin{enumerate}
%		\item
%			Show that the relative error in calculating $\frac{y}{x}$ satisfies
%			\begin{align*}
%				\delta_{\frac{y}{x}} & \approx \delta_y - \delta_x + \delta_y \delta_x \\
%                                                     & \approx \delta_y - \delta_x
%			\end{align*}
%		\item
%			Is there a loss of significant digits when dividing by a small number?
%	\end{enumerate}
%\end{question}

\end{document}
