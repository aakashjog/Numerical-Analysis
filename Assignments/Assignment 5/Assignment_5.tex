\documentclass[fleqn, a4paper, 11pt, oneside]{amsart}
%\usepackage[top = 2cm, bottom = 1cm, left = 1cm, right = 1cm]{geometry}
\usepackage{exsheets, tasks}
\usepackage{amsmath, amssymb, amsthm} %standard AMS packages
\usepackage{marginnote} %marginnotes
\usepackage{gensymb} %miscellaneous symbols
\usepackage{commath} %differential symbols
\usepackage{xcolor} %colours
\usepackage{cancel} %cancelling terms
\usepackage[free-standing-units, space-before-unit]{siunitx} %formatting units
\usepackage{tikz, pgfplots} %diagrams
\usetikzlibrary{calc, hobby, patterns, intersections, decorations.markings}
\usepackage{graphicx} %inserting graphics
\usepackage{hyperref} %hyperlinks
\usepackage{datetime} %date and time
\usepackage{ulem} %underline for \emph{}
\usepackage{xfrac} %inline fractions
\usepackage{enumerate,enumitem} %numbered lists
\usepackage{float} %inserting floats
\usepackage{circuitikz}[american voltages, american currents] %circuit diagrams

\newcommand\numberthis{\addtocounter{equation}{1}\tag{\theequation}} %adds numbers to specific equations in non-numbered list of equations

\newcommand{\AxisRotator}[1][rotate=0]{
	\tikz [x=0.25cm,y=0.60cm,line width=.2ex,-stealth,#1] \draw (0,0) arc (-150:150:1 and 1);%
} %rotation symbols on axes

\theoremstyle{definition}
\newtheorem{example}{Example}
\newtheorem{definition}{Definition}

\theoremstyle{theorem}
\newtheorem{theorem}{Theorem}

\newcommand{\curl}{\mathrm{curl\,}}

\makeatletter
\@addtoreset{section}{part} %resets section numbers in new part
\makeatother

\renewcommand{\thesubsection}{(\arabic{subsection})}
\renewcommand{\thesection}{(\arabic{section})}

%section headings on left
\makeatletter
\def\specialsection{\@startsection{section}{1}%
	\z@{\linespacing\@plus\linespacing}{.5\linespacing}%
	%  {\normalfont\centering}}% DELETED
	{\normalfont}}% NEW
\def\section{\@startsection{section}{1}%
	\z@{.7\linespacing\@plus\linespacing}{.5\linespacing}%
	%  {\normalfont\scshape\centering}}% DELETED
	{\normalfont\scshape}}% NEW
\makeatother

%forces newline after subsection
\makeatletter
\def\subsection{\@startsection{subsection}{3}%
	\z@{.5\linespacing\@plus.7\linespacing}{.1\linespacing}%
	{\normalfont\itshape}}
\makeatother

\settasks{counter-format = tsk[1].}

\SetupExSheets{solution/print = true}

%opening
\title{Numerical Analysis : Assignment 4}
\author
{
	Aakash Jog\\
	ID : 989323563
}
\date{\formatdate{10}{11}{2015}}

\begin{document}

\tikzset{->-/.style={decoration={
  markings,
  mark=at position #1 with {\arrow{>}}},postaction={decorate}}}

\maketitle
%\setlength{\mathindent}{0pt}

\begin{question}
	Consider the stopping criterion
	\begin{align*}
		\left| \frac{f(x_n)}{f'(x_n)} \right| & < \varepsilon
	\end{align*}
	Show that in Newton's method this criterion is equivalent to
	\begin{align*}
		|x_{n + 1} - x_n| & < \varepsilon
	\end{align*}
\end{question}

\begin{solution}
	\begin{align*}
		\left| \frac{f(x_n)}{f'(x_n)} \right| & < \varepsilon
	\end{align*}
	By Newton's method,
	\begin{align*}
		x_{n + 1}                    & = x_n - \frac{f(x_n)}{f'(x_n)}          \\
		\therefore x_{n + 1} - x_n   & = -\frac{f(x_n)}{f'(x_n)}               \\
		\therefore |x_{n + 1} - x_n| & = \left| \frac{f(x_n)}{f'(x_n)} \right| \\
		\therefore |x_{n + 1} - x_n| & \le \varepsilon
	\end{align*}
	\qed
\end{solution}

\begin{question}
	Let $x_{n + 1} = g(x_n)$ be a fixed point method, and assume that $x_n \to a$ as $n \to \infty$, where $g(a) = a$.
	\begin{enumerate}
		\item
			Let $e_n = a - x_n$.
			Show
			\begin{align*}
				x_{n + 1} - x_n & = e_n - e_{n + 1}
			\end{align*}
		\item
			Show that
			\begin{align*}
				e_{n + 1} & = g'(c_n) e_n
			\end{align*}
			for some $c_n$ between $a$ and $x_n$.
			Hint: Use the mean value theorem.
		\item
			Show that
			\begin{align*}
				e_n & \approx \frac{x_{n + 1} - x_n}{1 - g'(x_n)}
			\end{align*}
			and formulate a corresponding stopping criterion.
		\item
			Formulate a stopping criterion based on the above.
		\item
			When is the stopping criterion
			\begin{align*}
				|x_{n + 1} - x_n| & < \varepsilon
			\end{align*}
			a good approximation to the theoretical stopping criterion $|e_n| < \varepsilon$?
	\end{enumerate}
\end{question}

\begin{solution}
	\begin{enumerate}[leftmargin=*]
		\item
			\begin{align*}
				x_{n + 1} - x_n & = (a - e_{n + 1}) - (a - e_n) \\
                                                & = e_n - e_{n + 1}
			\end{align*}
			\qed
		\item
			By Lagrange's Mean Value theorem, $\exists c_n \in (a,x_n)$, such that
			\begin{align*}
				g'(c_n)              & = \frac{g(x_n) - g(a)}{x_n - a} \\
                                                     & = \frac{x_{n + 1} - a}{x_n - a} \\
                                                     & = \frac{e_{n + 1}}{e_n}         \\
				\therefore e_{n + 1} & = g'(c_n) e_n
			\end{align*}
			\qed
		\item
			\begin{align*}
				g'(c_n)                & = \frac{g(a) - g(x_n)}{a - x_n}                    \\
				\therefore 1 - g'(c_n) & = 1 - \left( \frac{g(a) - g(x_n)}{a - x_n} \right) \\
				\therefore 1 - g'(c_n) & = 1 - \left( \frac{g(a) - g(x_n)}{e_n} \right)     \\
				\therefore e_n(x)      & = \frac{x_{n + 1} - x_n}{1 - g'(x_n)}              \\
			\end{align*}
		\item
			The corresponding stopping criterion is
			\begin{align*}
				|e_n| & < \varepsilon
			\end{align*}
		\item
			If $g'(a) = 0$ or $g'(a) = 2$,
			\begin{align*}
				\left| e_n(x) \right| & = \left| \frac{x_{n + 1} - x_n}{1} \right|
			\end{align*}
			Therefore, this is a good approximation to the theoretical stopping criterion.
	\end{enumerate}
\end{solution}

\begin{question}
	Let
	\begin{align*}
		f(x) & = e^{-x} - \frac{1}{2}
	\end{align*}
	\begin{enumerate}
		\item
			Show that $f$ has a root in $[0,1]$.
		\item
			Show that Newton's method converges to the root $a$ of $f$.
		\item
			\begin{enumerate}
				\item How many points are required in order to guarantee an absolute error bounded by $10^{-2}$?
				\item How many points are required in order to guarantee an absolute error bounded by $10^{-7}$?
			\end{enumerate}
		\item
			How many iterations are performed if the above stopping criterion, to guarantee an error bounded by $10^{-2}$ and $10^{-7}$?
		\item
			Cheat and calculate $\ln(2)$, i.e. the root of $f$.
			What is a better estimate for the required number of iterations, the worst case bound of section 3 or the error estimate of section 4?
			Explain why one of them is better than the other.
			Base you explanation on the notion of the rate of convergence.
	\end{enumerate}
\end{question}

\begin{solution}
	\begin{enumerate}[leftmargin=*]
		\item
			\begin{align*}
				f(0) & = 1 - \frac{1}{2} \\
                                     & > 0
			\end{align*}
			\begin{align*}
				f(1) & = \frac{1}{e} - \frac{1}{2} \\
                                     & < 0
			\end{align*}
			Therefore, by the intermediate value theorem, $\exists c \in [0,1]$ such that $f(c) = 0$.
			Hence $f$ has a root in $[0,1]$.
			\qed
		\item
			\begin{align*}
				g(x) & = x - \frac{f(x)}{f'(x)} \\
                                     & = x + 1 - \frac{e^x}{2}
			\end{align*}
			As $g'(x) = 1 - \frac{e^x}{2}$, $g : [0,1] \to [0,1]$.
			Therefore, as $n \to \infty$, $x_n \to a$.
			\qed
		\item
			\begin{enumerate}[leftmargin=*]
				\item
					By the fixed point theorem,
					\begin{align*}
						|e_n| & \le \left( \frac{1}{2} \right)^2 |e_0| \\
                                                      & \le 2^{-n}                             \\
					\end{align*}
					Therefore,
					\begin{align*}
						10^{-2}         & \ge 2^{-n}        \\
						\therefore 10^2 & \le 2^n           \\
						\therefore n    & \ge 2 \log_{2} 10 \\
						\therefore n    & \ge 7
					\end{align*}
					Therefore, $7$ points are required in order to guarantee an error bounded by $10^{-2}$.
				\item
					\begin{align*}
						10^{-7}         & \ge 2^{-n}\\
						\therefore 10^7 & \le 2^n\\
						\therefore n    & \ge 7 \log_{2} 10\\
						\therefore n    & \ge 24
					\end{align*}
					Therefore, $24$ points are required in order to guarantee an error bounded by $10^{-7}$.
			\end{enumerate}
		\item
			\begin{align*}
				x_0 & = 0.5       \\
				x_1 & = 0.6756393 \\
				x_2 & = 0.6929948 \\
				x_3 & = 0.6931471 \\
				x_4 & = 0.6931471 \\
			\end{align*}
			Therefore, to guarantee an error bounded by $10^{-2}$ and $10^{-7}$, $2$ and $3$ iterations, respectively, are required.
		\item
			\begin{align*}
				\ln(2) & = 0.6931471
			\end{align*}
			Therefore, the error estimate of section 4 is better, as for it, $n = 3$ and $\varepsilon = 10^{-7}$.
			In Newton's method, $g'(a) = 0$.
			Hence, the convergence is expected to be faster.
	\end{enumerate}
\end{solution}

\begin{question}
	We want to approximate a root $a$ of $f(x) = x - e^{-x^2}$, in $[0,1]$.
	\begin{enumerate}
		\item
			Show that there exists such a root.
		\item
			We define the iterative method
			\begin{align*}
				x_{n + 1} & = g(x_n) \\
                                          & = e^{-{x_n}^2}
			\end{align*}
			Show that any root of $f$ is a fixed point of $g$ and vice-versa.
		\item
			Show that the root of $f$ is unique in $[0,1]$, and that the iterative method converges to it.
		\item
			Let $x_0 \in [0,1]$ be an initial guess.
			How many iterations are required in order to guarantee an error bounded by 0.05?
	\end{enumerate}
\end{question}

\begin{solution}
	\begin{enumerate}[leftmargin=*]
		\item
			\begin{align*}
				f(0) & = 0 - 1 \\
                                     & < 0
			\end{align*}
			\begin{align*}
				f(1) & = 1 - \frac{1}{e} \\
                                     & > 0
			\end{align*}
			Therefore, by the intermediate value theorem, there exists such a root $a$ in $[0,1]$.
		\item
			\begin{align*}
				x_{n + 1} & = g(x_n) \\
                                          & = e^{-{x_n}^2}
			\end{align*}
			Therefore,
			\begin{align*}
				g(x)              & = x        \\
				\iff x            & = e^{-x^2} \\
				\iff x - e^{-x^2} & = 0        \\
				\iff f(x)         & = 0
			\end{align*}
		\item
			\begin{align*}
				g(x)             & = e^{-x^2} \\
				\therefore g'(x) & = -2 x e^{-x^2}
			\end{align*}
			Therefore,
			\begin{align*}
				g(0) & = 1           \\
                                     & \le 1         \\
				g(1) & = \frac{1}{e} \\
                                     & \le 1
			\end{align*}
			\begin{align*}
				\left| g'(x) \right| & = \left| -2 x e^{-x^2} \right| \\
                                                     & = 2 x e^{-x^2}                 \\
                                                     & \le \sqrt{\frac{2}{e}}         \\
                                                     & < 1
			\end{align*}
			Therefore, by the fixed point theorem, $x_n \to a$, as $n \to \infty$, and $a$ is unique.
		\item
			\begin{align*}
				|e_n| & \le k^n |e_0|                                 \\
                                      & \le \left( \sqrt{\frac{2}{e}} \right)^n |e_0| \\
                                      & \le \left( \sqrt{\frac{2}{e}} \right)^n
			\end{align*}
			Therefore,
			\begin{align*}
				\left( \sqrt{\frac{2}{e}} \right)^n & < 5 \times 10^{-2}
				\therefore n                        & \ge 20
			\end{align*}
			Therefore, at least 20 iterations are required to guarantee an error bounded by 0.05.
	\end{enumerate}
\end{solution}

\end{document}
