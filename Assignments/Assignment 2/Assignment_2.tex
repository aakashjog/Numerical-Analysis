\documentclass[fleqn, a4paper, 11pt, oneside]{amsart}
%\usepackage[top = 2cm, bottom = 1cm, left = 1cm, right = 1cm]{geometry}
\usepackage{exsheets, tasks}
\usepackage{amsmath, amssymb, amsthm} %standard AMS packages
\usepackage{marginnote} %marginnotes
\usepackage{gensymb} %miscellaneous symbols
\usepackage{commath} %differential symbols
\usepackage{xcolor} %colours
\usepackage{cancel} %cancelling terms
\usepackage[free-standing-units, space-before-unit]{siunitx} %formatting units
\usepackage{tikz, pgfplots} %diagrams
\usetikzlibrary{calc, hobby, patterns, intersections, decorations.markings}
\usepackage{graphicx} %inserting graphics
\usepackage{hyperref} %hyperlinks
\usepackage{datetime} %date and time
\usepackage{ulem} %underline for \emph{}
\usepackage{xfrac} %inline fractions
\usepackage{enumerate,enumitem} %numbered lists
\usepackage{float} %inserting floats
\usepackage{circuitikz}[american voltages, american currents] %circuit diagrams

\newcommand\numberthis{\addtocounter{equation}{1}\tag{\theequation}} %adds numbers to specific equations in non-numbered list of equations

\newcommand{\AxisRotator}[1][rotate=0]{
	\tikz [x=0.25cm,y=0.60cm,line width=.2ex,-stealth,#1] \draw (0,0) arc (-150:150:1 and 1);%
} %rotation symbols on axes

\theoremstyle{definition}
\newtheorem{example}{Example}
\newtheorem{definition}{Definition}

\theoremstyle{theorem}
\newtheorem{theorem}{Theorem}

\newcommand{\curl}{\mathrm{curl\,}}

\makeatletter
\@addtoreset{section}{part} %resets section numbers in new part
\makeatother

\renewcommand{\thesubsection}{(\arabic{subsection})}
\renewcommand{\thesection}{(\arabic{section})}

%section headings on left
\makeatletter
\def\specialsection{\@startsection{section}{1}%
	\z@{\linespacing\@plus\linespacing}{.5\linespacing}%
	%  {\normalfont\centering}}% DELETED
	{\normalfont}}% NEW
\def\section{\@startsection{section}{1}%
	\z@{.7\linespacing\@plus\linespacing}{.5\linespacing}%
	%  {\normalfont\scshape\centering}}% DELETED
	{\normalfont\scshape}}% NEW
\makeatother

%forces newline after subsection
\makeatletter
\def\subsection{\@startsection{subsection}{3}%
	\z@{.5\linespacing\@plus.7\linespacing}{.1\linespacing}%
	{\normalfont\itshape}}
\makeatother

\settasks{counter-format = tsk[1].}

\SetupExSheets{solution/print = true}

%opening
\title{Numerical Analysis : Assignment 2}
\author
{
	Aakash Jog\\
	ID : 989323563
}
\date{\formatdate{27}{10}{2015}}

\begin{document}

\tikzset{->-/.style={decoration={
  markings,
  mark=at position #1 with {\arrow{>}}},postaction={decorate}}}

\maketitle
%\setlength{\mathindent}{0pt}

\begin{question}
	\begin{enumerate}
		\item
			Compute the interpolating polynomial for $f(x) = \sqrt{x}$, that interpolates $f$ at the sample points $\left\{ \frac{1}{4} , 1 , 4 \right\}$.
		\item
			Compute the interpolating polynomial for $g(x) = 2^x$, that interpolates $g$ at the sample points $\{-1,0,1\}$.
		\item
			Approximate the number $\sqrt{2}$ in two ways.
			First by interpolating polynomial of $f$ from $1$, and then by the interpolating polynomial of $g$ from $2$.
	\end{enumerate}
\end{question}

\begin{solution}
	\begin{enumerate}[leftmargin=*]
		\item
			Let
			\begin{align*}
				p_f(x) &= a_0 + a_1 x + a_2 x^2
			\end{align*}
			Therefore,
			\begin{align*}
				p_f\left( \frac{1}{4} \right) &= a_0 + \frac{1}{4} a_1 + \frac{1}{16} a_2\\
				\therefore \frac{1}{2} &= a_0 + \frac{1}{4} a_1 + \frac{1}{16} a_2\\
				p_f(1) &= a_0 + a_1 + a_2\\
				\therefore 1 &= a_0 + a_1 + a_2\\
				p_f(4) &= a_0 + 4 a_1 + 16 a_2\\
				\therefore 2 &= a_0 + 4 a_1 + 16 a_2
			\end{align*}
			Therefore,
			\begin{align*}
					\begin{pmatrix}
						1 & \frac{1}{4} & \frac{1}{16}\\
						1 & 1 & 1\\
						1 & 4 & 16\\
					\end{pmatrix}
					\begin{pmatrix}
						a_0\\
						a_1\\
						a_2\\
					\end{pmatrix}
				&=
					\begin{pmatrix}
						\frac{1}{2}\\
						1\\
						2\\
					\end{pmatrix}
			\end{align*}
			Therefore,
			\begin{align*}
					\begin{pmatrix}
						a_0\\
						a_1\\
						a_2\\
					\end{pmatrix}
				&=
					\begin{pmatrix}
						\frac{14}{45}\\
						\frac{7}{9}\\
						-\frac{4}{45}\\
					\end{pmatrix}
			\end{align*}
			Therefore,
			\begin{align*}
				p_f(x) &= \frac{14}{45} + \frac{7}{9} x - \frac{4}{45} x^2
			\end{align*}
		\item
			Let
			\begin{align*}
				p_g(x) &= b_0 + b_1 x + b_2 x^2
			\end{align*}
			Therefore,
			\begin{align*}
				p_g(-1) &= b_0 - b_1 + b_2\\
				\therefore \frac{1}{2} &= b_0 - b_1 + b_2\\
				p_g(0) &= b_0\\
				\therefore 1 &= b_0\\
				p_g(1) &= b_0 + b_1 + b_2\\
				\therefore 2 &= b_0 + b_1 + b_2
			\end{align*}
			Therefore,
			\begin{align*}
					\begin{pmatrix}
						1 & -1 & 1\\
						1 & 0 & 0\\
						1 & 1 & 1\\
					\end{pmatrix}
					\begin{pmatrix}
						b_0\\
						b_1\\
						b_2\\
					\end{pmatrix}
				&=
					\begin{pmatrix}
						\frac{1}{2}\\
						1\\
						2\\
					\end{pmatrix}
			\end{align*}
			Therefore,
			\begin{align*}
					\begin{pmatrix}
						b_0\\
						b_1\\
						b_2\\
					\end{pmatrix}
				&=
					\begin{pmatrix}
						1\\
						\frac{3}{4}\\
						\frac{1}{4}\\
					\end{pmatrix}
			\end{align*}
			Therefore,
			\begin{align*}
				p_g(x) &= 1 + \frac{3}{4} x + \frac{1}{4} x^2
			\end{align*}
		\item
			\begin{align*}
				p_f(x) &= \frac{14}{45} + \frac{7}{9} x - \frac{4}{45} x^2\\
				\therefore \sqrt{2} &= \frac{14}{45} + \frac{7}{9} \cdot 2 - \frac{4}{45} \cdot 4\\
				&= \frac{14}{45} + \frac{14}{9} - \frac{16}{45}\\
				&= \frac{14 + 70 - 16}{45}\\
				&= \frac{68}{45}
			\end{align*}
			\begin{align*}
				p_g(x) &= 1 + \frac{3}{4} x + \frac{1}{4} x^2\\
				\therefore \sqrt{2} &= 1 + \frac{3}{4} \cdot \frac{1}{2} + \frac{1}{4} \cdot \frac{1}{4}\\
				&= 1 + \frac{3}{8} + \frac{1}{16}\\
				&= \frac{16 + 6 + 1}{16}\\
				&= \frac{23}{16}
			\end{align*}
	\end{enumerate}
\end{solution}

\begin{question}
	Consider the interpolation data
	\begin{table}[H]
		\centering
		\begin{tabular}{|c|c|}
			\hline
			$x_i$ & $y_i$\\
			\hline
			$-1$ & $-1$\\
			$0$ & $2$\\
			$1$ & $5$\\
			\hline
		\end{tabular}
	\end{table}
	\begin{enumerate}
		\item
			Find the interpolating polynomial using Lagrange polynomials.
		\item
			Find the interpolating polynomial using Newton's method.
		\item
			Is the polynomial you got of degree 2?
			Does this contradict the uniqueness and existence of the interpolating polynomial?
	\end{enumerate}
\end{question}

\begin{solution}
	\begin{enumerate}[leftmargin=*]
		\item
			\begin{align*}
				x_0 &= -1\\
				x_1 &= 0\\
				x_2 &= 1
			\end{align*}
			Therefore,
			\begin{align*}
				l_0(x) &= \frac{(x - x_1) (x - x_2)}{(x_0 - x_1) (x_0 - x_2)}\\
				&= \frac{(x - 0) (x - 1)}{(-1 - 0) (-1 - 1)}\\
				&= \frac{(x) (x - 1)}{2}\\
				l_1(x) &= \frac{(x - x_0) (x - x_2)}{(x_1 - x_0) (x_1 - x_2)}\\
				&= \frac{(x + 1) (x - 1)}{(0 + 1) (0 - 1)}\\
				&= -(x + 1) (x - 1)\\
				l_2(x) &= \frac{(x - x_0) (x - x_1)}{(x_2 - x_0) (x_2 - x_1)}\\
				&= \frac{(x + 1) (x - 0)}{(1 + 1) (1 - 0)}\\
				&= \frac{(x) (x + 1)}{2}
			\end{align*}
			Therefore,
			\begin{align*}
				p(x) &= f(x_0) l_0(x) + f(x_1) l_1(x) + f(x_2) l_2(x)\\
				&= -\frac{(x) (x - 1)}{2} - 2 (x + 1) (x - 1) + 5 \frac{(x) (x + 1)}{2}\\
				&= -\frac{x^2 - x}{2} - 2 (x^2 - 1) + 5 \frac{x^2 + x}{2}\\
				&= \frac{4 x^2 + 6 x}{2} - 2 x^2 + 2\\
				&= 2 x^2 + 3 x - 2 x^2 + 2\\
				&= 3 x + 2
			\end{align*}
		\item

		\item
			The polynomial is of degree 1.
			However this does not contradict the uniqueness and existence of the interpolating polynomial, as all points $(x_i,y_i)$ are collinear.
			Hence, the polynomial represents a line through these points, and not a parabola.
			If one of the points would not have lied on the straight line defined by the other two points, the interpolating polynomial would have been of degree 2.
	\end{enumerate}
\end{solution}

\begin{question}
	Let $x_0$, $x_1$, $x_2$ be three points.
	We want to approximate the function $f(x)$ using a polynomial $p$ of degree up to 2 that satisfies
	\begin{align*}
		p(x_0) &= f(x_0)\\
		p'(x_1) &= f'(x_1)\\
		p(x_2) &= f(x_2)
	\end{align*}
	Assume that $x_0 \neq x_2$, and prove that $p(x)$ exists and is unique if and only if
	\begin{align*}
		x_1 &\neq \frac{x_0 + x_2}{2}
	\end{align*}
	Hint: Note that this is not the standard interpolation problem.
	Write $p(x)$ in its standard form, and derive a system of linear equations with polynomial coefficients as the unknowns.
	For a system of linear equations, what is the condition which is equivalent to existence and uniqueness of a solution?
\end{question}

\begin{solution}
	Let
	\begin{align*}
		p(x) &= a_0 + a_1 x + a_2 x^2
	\end{align*}
	Therefore,
	\begin{align*}
		p(x_0) &= a_0 + a_1 x_0 + a_2 {x_0}^2\\
		p(x_1) &= a_0 + a_1 x_1 + a_2 {x_1}^2\\
		p(x_2) &= a_0 + a_1 x_2 + a_2 {x_2}^2\\
	\end{align*}
	Therefore,
	\begin{align*}
		V &=
			\begin{pmatrix}
				1 & x_0 & {x_0}^2\\
				1 & x_1 & {x_1}^2\\
				1 & x_2 & {x_2}^2\\
			\end{pmatrix}
	\end{align*}
	Therefore, as the Van der Monde matrix must be invertible,
	\begin{align*}
		|V| &\neq 0\\
		\therefore 0 &\neq \left( x_1 {x_2}^2 - x_2 {x_1}^2 \right) - x_0 \left( {x_2}^2 - {x_1}^2 \right) + {x_0}^2 (x_2 - x_1)\\
		\therefore 0 &\neq x_1 {x_2}^2 - x_2 {x_1}^2 - x_0 {x_2}^2 + x_0 {x_1}^2 + {x_0}^2 x_2 - {x_0}^2 x_1\\
		&= (x_0 - x_1) (x_1 - x_2) (x_2 - x_0)
	\end{align*}
	As $x_0 \neq x_2$,
	\begin{align*}
		(x_2 - x_0) &\neq 0
	\end{align*}
	Therefore,
	\begin{align*}
		|V| &\neq 0\\
		\iff (x_0 - x_1) (x_1 - x_2) &\neq 0
	\end{align*}
	Therefore,
	\begin{align*}
		x_1 &\neq \frac{x_0 + x_2}{2}
	\end{align*}
	\qed
\end{solution}

\begin{question}
	Let $l_k(x)$, $k = 0,\dots,n$ be the Lagrange polynomials with respect to the points $x_0,\dots,x_n$.
	Show
	\begin{enumerate}
		\item
			for any $x \in \mathbb{R}$,
			\begin{align*}
				\sum\limits_{k = 0}^{n} l_k(x) &= 1
			\end{align*}
		\item
			for any $x \in \mathbb{R}$ and $m = 0,\dots,n$,
			\begin{align*}
				\sum\limits_{k = 0}^{n} {x_k}^m l_k(x) &= x^m
			\end{align*}
	\end{enumerate}
\end{question}

\begin{solution}
	\begin{enumerate}[leftmargin=*]
		\item
			\begin{align*}
				l_k(x) &= \frac{(x - x_0) \dots (x - x_{k - 1}) (x - x_{k + 1}) \dots (x - x_n)}{(x_k - x_0) \dots (x_k - x_{k - 1}) (x_k - x_{k + 1}) \dots (x_k - x_n)}
			\end{align*}
			Therefore,
			\begin{align*}
				p(x) &= \sum\limits_{k = 1}^{n} f(x_k) l_k(x)\\
			\end{align*}
			Therefore, if $p(x) = f(x) = 1$,
			\begin{align*}
				1 &= \sum\limits_{k = 1}^{n} l_k(x)
			\end{align*}
		\item
			By definition,
			\begin{align*}
				l_k(x_i) &=
					\begin{cases}
						1 &\quad k = i\\
						0 &\quad k \neq i\\
					\end{cases}
			\end{align*}
			Therefore,
			\begin{align*}
				{x_k}^m l_k(x_i) &=
					\begin{cases}
						{x_k}^m &\quad k = i\\
						0 &\quad k \neq i\\
					\end{cases}
			\end{align*}
			Also,
			\begin{align*}
				\sum\limits_{k = 0}^{n} l_k(x) &= 1
			\end{align*}
			Therefore,
			\begin{align*}
				\sum\limits_{k = 0}^{n} {x_k}^m l_k(x) &= x^m
			\end{align*}
	\end{enumerate}
\end{solution}

\end{document}
