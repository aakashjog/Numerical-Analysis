\documentclass[fleqn, a4paper, 11pt, oneside]{amsart}
%\usepackage[top = 2cm, bottom = 1cm, left = 1cm, right = 1cm]{geometry}
\usepackage{exsheets, tasks}
\usepackage{amsmath, amssymb, amsthm} %standard AMS packages
\usepackage{marginnote} %marginnotes
\usepackage{gensymb} %miscellaneous symbols
\usepackage{commath} %differential symbols
\usepackage{xcolor} %colours
\usepackage{cancel} %cancelling terms
\usepackage[free-standing-units, space-before-unit]{siunitx} %formatting units
\usepackage{tikz, pgfplots} %diagrams
\usetikzlibrary{calc, hobby, patterns, intersections, decorations.markings}
\usepackage{graphicx} %inserting graphics
\usepackage{hyperref} %hyperlinks
\usepackage{datetime} %date and time
\usepackage{ulem} %underline for \emph{}
\usepackage{xfrac} %inline fractions
\usepackage{enumerate,enumitem} %numbered lists
\usepackage{float} %inserting floats
\usepackage{circuitikz}[american voltages, american currents] %circuit diagrams

\newcommand\numberthis{\addtocounter{equation}{1}\tag{\theequation}} %adds numbers to specific equations in non-numbered list of equations

\newcommand{\AxisRotator}[1][rotate=0]{
	\tikz [x=0.25cm,y=0.60cm,line width=.2ex,-stealth,#1] \draw (0,0) arc (-150:150:1 and 1);%
} %rotation symbols on axes

\theoremstyle{definition}
\newtheorem{example}{Example}
\newtheorem{definition}{Definition}

\theoremstyle{theorem}
\newtheorem{theorem}{Theorem}

\newcommand{\curl}{\mathrm{curl\,}}

\makeatletter
\@addtoreset{section}{part} %resets section numbers in new part
\makeatother

\renewcommand{\thesubsection}{(\arabic{subsection})}
\renewcommand{\thesection}{(\arabic{section})}

\DeclareMathOperator{\cond}{cond}

%section headings on left
\makeatletter
\def\specialsection{\@startsection{section}{1}%
	\z@{\linespacing\@plus\linespacing}{.5\linespacing}%
	%  {\normalfont\centering}}% DELETED
	{\normalfont}}% NEW
\def\section{\@startsection{section}{1}%
	\z@{.7\linespacing\@plus\linespacing}{.5\linespacing}%
	%  {\normalfont\scshape\centering}}% DELETED
	{\normalfont\scshape}}% NEW
\makeatother

%forces newline after subsection
\makeatletter
\def\subsection{\@startsection{subsection}{3}%
	\z@{.5\linespacing\@plus.7\linespacing}{.1\linespacing}%
	{\normalfont\itshape}}
\makeatother

\settasks{counter-format = tsk[1].}

\SetupExSheets{solution/print = true}

%opening
\title{Numerical Analysis : Assignment 8}
\author
{
	Aakash Jog\\
	ID : 989323563
}
\date{\formatdate{15}{12}{2015}}

\begin{document}

\tikzset{->-/.style={decoration={
  markings,
  mark=at position #1 with {\arrow{>}}},postaction={decorate}}}

\maketitle
%\setlength{\mathindent}{0pt}

\begin{question}
	Let
	\begin{align*}
		A &=
			\begin{pmatrix}
				1 & 2 \\
				3 & 4 \\
			\end{pmatrix}\\
		B &=
			\begin{pmatrix}
				3 & 1 & 0 \\
				1 & 3 & 1 \\
				0 & 1 & 3 \\
			\end{pmatrix}
	\end{align*}
	Calculate the condition numbers of $A$ and $B$ with respect to the $1$, $2$, and $\infty$ norms,
\end{question}

\begin{solution}
	\begin{align*}
		A^{-1} &=
			\begin{pmatrix}
				-2  & 1    \\
				1.5 & -0.5 \\
			\end{pmatrix}
	\end{align*}
	Therefore,
	\begin{align*}
		\|A\|_1        & = \max\left\{ (1 + 3) , (2 + 4) \right\}     \\
                               & = 6                                          \\
		{\|A\|^{-1}}_1 & = \max\left\{ (2 + 1.5) , (1 + 0.5) \right\} \\
                               & = 3.5
	\end{align*}
	Therefore,
	\begin{align*}
		\cond(A)_1 & = \|A\|_1 \left\| A^{-1} \right\|_1 \\
                           & = 21
	\end{align*}
	~\\
	\begin{align*}
		A^{\mathsf{T}} &= 
			\begin{pmatrix}
				1 & 3 \\
				2 & 4 \\
			\end{pmatrix}\\
		\therefore A^{\mathsf{T}} A &=
			\begin{pmatrix}
				10 & 14 \\
				14 & 20 \\
			\end{pmatrix}
	\end{align*}
	Therefore,
	\begin{align*}
		\rho(A)            & = \frac{3345}{112}        \\
		\therefore \|A\|_2 & = \sqrt{\frac{3345}{112}} \\
                                   & = 5.4650
	\end{align*}
	Therefore,
	\begin{align*}
		{A^{-1}}^{\mathsf{T}} &= 
			\begin{pmatrix}
				-2 & 1.5  \\
				1  & -0.5 \\
			\end{pmatrix}\\
		\therefore {A^{-1}}^{\mathsf{T}} A^{-1} &=
			\begin{pmatrix}
				6.25  & -2.75 \\
				-2.75 & 1.25  \\
			\end{pmatrix}
	\end{align*}
	Therefore,
	\begin{align*}
		\rho\left( A^{-1} \right) & = \frac{3345}{448}        \\
		\therefore \|A\|_2        & = \sqrt{\frac{3345}{448}} \\
                                          & = 2.7325
	\end{align*}
	Therefore,
	\begin{align*}
		\cond(A)_2 & = \|A\|_2 \left\| A^{-1} \right\|_2 \\
                           & = 14.9331125
	\end{align*}
	~\\
	\begin{align*}
		\|A\|_{\infty}        & = \max\left\{ (1 + 2) , (3 + 4) \right\}     \\
                                      & = 7                                          \\
		{\|A\|^{-1}}_{\infty} & = \max\left\{ (2 + 1) , (1.5 + 0.5) \right\} \\
                                      & = 3
	\end{align*}
	Therefore,
	\begin{align*}
		\cond(A)_{\infty} & = \|A\|_{\infty} \left\| A^{-1} \right\|_{\infty} \\
                                  & = 21
	\end{align*}
	~\\
	\begin{align*}
		B^{-1} &=
			\begin{pmatrix}
				\frac{8}{21}  & -\frac{3}{21} & \frac{1}{21}  \\
				-\frac{3}{21} & \frac{9}{21}  & -\frac{3}{21} \\
				\frac{1}{21}  & -\frac{3}{21} & \frac{8}{21}  \\
			\end{pmatrix}
	\end{align*}
	Therefore,
	\begin{align*}
		\|B\|_1                   & = 5 \\
		\left\| B^{-1} \right\|_1 & = \frac{5}{7}
	\end{align*}
	Therefore,
	\begin{align*}
		\cond(B)_1 & = \frac{25}{7}
	\end{align*}
	Therefore,
	\begin{align*}
		\|B\|_2                   & = 3 + \sqrt{2} \\
		\left\| B^{-1} \right\|_2 & = \frac{3 + \sqrt{17}}{14}
	\end{align*}
	Therefore,
	\begin{align*}
		\cond(B)_1 & = \frac{\left( 3 + \sqrt{2} \right) \left( 3 + \sqrt{17} \right)}{14}
	\end{align*}
	Therefore,
	\begin{align*}
		\|B\|_{\infty}                   & = 5 \\
		\left\| B^{-1} \right\|_{\infty} & = \frac{5}{7}
	\end{align*}
	Therefore,
	\begin{align*}
		\cond(B)_{\infty} & = \frac{25}{7}
	\end{align*}
\end{solution}

\begin{question}
	Let
	\begin{align*}
		A &=
			\begin{pmatrix}
				100 & 99 \\
				99  & 98 \\
			\end{pmatrix}\\
		b &=
			\begin{pmatrix}
				1.005 \\
				0.995 \\
			\end{pmatrix}\\
		e_b &=
			\begin{pmatrix}
				0.05  \\
				-0.05 \\
			\end{pmatrix}
	\end{align*}
	Let $x$ be the solution to the system
	\begin{align*}
		A x & = b
	\end{align*}
	Assume that the right hand side of the system is noisy, and we actually solve the system
	\begin{align*}
		A \tilde{x} & = \tilde{b}
	\end{align*}
	where
	\begin{align*}
		\tilde{b} & = b - e_b
	\end{align*}
	Note that this is not the question from class.
	In class we assumed that the RHS is accurate, but the solution to the system is calculated inaccurately by a numerical method.
	In this exercise we assume that we know the RHS up to an error, and we calculate the accurate solution of the system with a noisy RHS.
	\begin{enumerate}
		\item
			Calculate the condition number of $A$ with respect to the $\infty$ norm.
		\item
			Calculate the relative error in the RHS, in the $\infty$ norm.
		\item
			Without solving the system, find a bound on the relative error in the solution $x$, in the $\infty$ norm.
		\item
			Calculate $x$ and $\tilde{x}$, and calculate the relative error, in the $\infty$, in the solution $x$.
			How does this error relate to the bound from 3?
	\end{enumerate}
\end{question}

\begin{solution}
	\begin{enumerate}[leftmargin=*]
		\item
			\begin{align*}
				A &=
					\begin{pmatrix}
						100 & 99 \\
						99  & 98 \\
					\end{pmatrix}\\
				\therefore A^{-1} &=
					\begin{pmatrix}
						-98 & 99   \\
						99  & -100 \\
					\end{pmatrix}
			\end{align*}
			Therefore,
			\begin{align*}
				\|A\|_{\infty}                   & = \max\left\{ (100 + 99) , (99 + 98) \right\} \\
                                                                 & = 199                                         \\
				\left\| A^{-1} \right\|_{\infty} & = \max\left\{ (98 + 99) , (99 + 100) \right\} \\
                                                                 & = 199
			\end{align*}
			Therefore,
			\begin{align*}
				\cond(A) & = 199^2 \\
                                         & = 39601
			\end{align*}
		\item
			The relative error in the RHS, in the $\infty$ norm, is
			\begin{align*}
				\frac{\|e_b\|_{\infty}}{\|b\|_{\infty}} & = \frac{0.05}{1.005} \\
                                                                        & \approx 0.05
			\end{align*}
		\item
			\begin{align*}
				\frac{\|e_x\|_{\infty}}{\|x\|_{\infty}} & \le \cond(A) \frac{\|e_b\|_{\infty}}{\|b\|_{\infty}} \\
                                                                        & = \left( 199^2 \right) \left( 0.05 \right)           \\
                                                                        & \approx 1980
			\end{align*}
		\item
			\begin{align*}
				x &= A^{-1} b\\
				&=
					\begin{pmatrix}
						-98 & 99   \\
						99  & -100 \\
					\end{pmatrix}
					\begin{pmatrix}
						1.005 \\
						0.995 \\
					\end{pmatrix}
			\end{align*}
			Therefore,
			\begin{align*}
				\tilde{x} &= A^{-1} \tilde{b}\\
				&= A^{-1} (b - e_b)\\
				&=
					\begin{pmatrix}
						-98 & 99   \\
						99  & -100 \\
					\end{pmatrix}
					\begin{pmatrix}
						1.005 - 0.05 \\
						0.995 + 0.05 \\
					\end{pmatrix}\\
				&=
					\begin{pmatrix}
						9.865  \\
						-9.955 \\
					\end{pmatrix}
			\end{align*}
			Therefore,
			\begin{align*}
				\frac{\left\| x - \tilde{x} \right\|_{\infty}}{\|x\|_{\infty}} & = 664
			\end{align*}
			Therefore, the error is approximately $\frac{1}{3}$ of the bound from 3.
	\end{enumerate}
\end{solution}

\begin{question}
	Let $A$ be a matrix, and $x_0$ be a vector.
	Consider the iterative method
	\begin{align*}
		x_{n + 1} & = A x_n
	\end{align*}
	\begin{enumerate}
		\item
			Prove
			\begin{align*}
				x_n & = A^n x_0
			\end{align*}
		\item
			Assume that $\|A\| < 1$ in some norm.
			Prove
			\begin{align*}
				\lim\limits_{n \to \infty} x_n & = \overrightarrow{0}
			\end{align*}
			You may use the proposition
			\begin{align*}
				\lim\limits_{n \to \infty} x_n          & = \overrightarrow{0} \\
				\iff \lim\limits_{n \to \infty} \|x_n\| & = 0
			\end{align*}
		\item
			Let
			\begin{align*}
				A &=
					\begin{pmatrix}
						0           & 2 \\
						\frac{1}{3} & 0 \\
					\end{pmatrix}
			\end{align*}
			Calculate the $1$, $2$, and $\infty$ norms of $A$.
			Can you use section 2, in order to prove or disprove $\lim\limits_{n \to \infty} x_n = \overrightarrow{0}$ ?
		\item
			Calculate $A^2$, and the $\infty$ norm of $A^2$.
		\item
			Use $A^2$ and $\left\| A^2 \right\|_{\infty}$ to show
			\begin{align*}
				\lim\limits_{n \to \infty} x_n & = \overrightarrow{0}
			\end{align*}
	\end{enumerate}
\end{question}

\begin{solution}
	\begin{enumerate}[leftmargin=*]
		\item
			\begin{align*}
				x_{n + 1} & = A x_n           \\
                                          & = A (A x_{n - 1}) \\
                                          & \vdots            \\
                                          & = A^n x_0
			\end{align*}
		\item
			\begin{align*}
				\lim\limits_{n \to \infty} \|x_n\| & = \lim\limits_{n \to \infty} \left\| A^n x_0 \right\| \\
                                                                   & = \lim\limits_{n \to \infty} \|A\|^n \|x_0\|          \\
                                                                   & = 0
			\end{align*}
			Therefore, by the proposition,
			\begin{align*}
				\lim\limits_{n \to \infty} x_n & = \overrightarrow{0}
			\end{align*}
		\item
			\begin{align*}
				A &=
					\begin{pmatrix}
						0           & 2 \\
						\frac{1}{3} & 0 \\
					\end{pmatrix}
			\end{align*}
			Therefore,
			\begin{align*}
				\|A\|_1 & = 2
			\end{align*}
			Therefore,
			\begin{align*}
				\|A\|_2 & = 2
			\end{align*}
			Therefore,
			\begin{align*}
				\|A\|_{\infty} & = 2
			\end{align*}
			As $\|A\| > 1$, section 2 cannot be used.
			Hence, the limit is zero if and only if $x_n = \overrightarrow{0}$.
		\item
			\begin{align*}
				A &=
					\begin{pmatrix}
						0           & 2 \\
						\frac{1}{3} & 0 \\
					\end{pmatrix}\\
				\therefore A^2 &=
					\begin{pmatrix}
						\frac{2}{3} & 0           \\
						0           & \frac{2}{3} \\
					\end{pmatrix}
			\end{align*}
			Therefore,
			\begin{align*}
				\left\| A^2 \right\|_{\infty} & = \frac{2}{3}
			\end{align*}
			Therefore, using section 2,
			\begin{align*}
				\lim\limits_{n \to \infty} x_n & = \overrightarrow{0}
			\end{align*}
	\end{enumerate}
\end{solution}

\end{document}
