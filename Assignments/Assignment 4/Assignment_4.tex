\documentclass[fleqn, a4paper, 11pt, oneside]{amsart}
%\usepackage[top = 2cm, bottom = 1cm, left = 1cm, right = 1cm]{geometry}
\usepackage{exsheets, tasks}
\usepackage{amsmath, amssymb, amsthm} %standard AMS packages
\usepackage{marginnote} %marginnotes
\usepackage{gensymb} %miscellaneous symbols
\usepackage{commath} %differential symbols
\usepackage{xcolor} %colours
\usepackage{cancel} %cancelling terms
\usepackage[free-standing-units, space-before-unit]{siunitx} %formatting units
\usepackage{tikz, pgfplots} %diagrams
\usetikzlibrary{calc, hobby, patterns, intersections, decorations.markings}
\usepackage{graphicx} %inserting graphics
\usepackage{hyperref} %hyperlinks
\usepackage{datetime} %date and time
\usepackage{ulem} %underline for \emph{}
\usepackage{xfrac} %inline fractions
\usepackage{enumerate,enumitem} %numbered lists
\usepackage{float} %inserting floats
\usepackage{circuitikz}[american voltages, american currents] %circuit diagrams

\newcommand\numberthis{\addtocounter{equation}{1}\tag{\theequation}} %adds numbers to specific equations in non-numbered list of equations

\newcommand{\AxisRotator}[1][rotate=0]{
	\tikz [x=0.25cm,y=0.60cm,line width=.2ex,-stealth,#1] \draw (0,0) arc (-150:150:1 and 1);%
} %rotation symbols on axes

\theoremstyle{definition}
\newtheorem{example}{Example}
\newtheorem{definition}{Definition}

\theoremstyle{theorem}
\newtheorem{theorem}{Theorem}

\newcommand{\curl}{\mathrm{curl\,}}

\makeatletter
\@addtoreset{section}{part} %resets section numbers in new part
\makeatother

\renewcommand{\thesubsection}{(\arabic{subsection})}
\renewcommand{\thesection}{(\arabic{section})}

%section headings on left
\makeatletter
\def\specialsection{\@startsection{section}{1}%
	\z@{\linespacing\@plus\linespacing}{.5\linespacing}%
	%  {\normalfont\centering}}% DELETED
	{\normalfont}}% NEW
\def\section{\@startsection{section}{1}%
	\z@{.7\linespacing\@plus\linespacing}{.5\linespacing}%
	%  {\normalfont\scshape\centering}}% DELETED
	{\normalfont\scshape}}% NEW
\makeatother

%forces newline after subsection
\makeatletter
\def\subsection{\@startsection{subsection}{3}%
	\z@{.5\linespacing\@plus.7\linespacing}{.1\linespacing}%
	{\normalfont\itshape}}
\makeatother

\settasks{counter-format = tsk[1].}

\SetupExSheets{solution/print = true}

%opening
\title{Numerical Analysis : Assignment 4}
\author
{
	Aakash Jog\\
	ID : 989323563
}
\date{\formatdate{10}{11}{2015}}

\begin{document}

\tikzset{->-/.style={decoration={
  markings,
  mark=at position #1 with {\arrow{>}}},postaction={decorate}}}

\maketitle
%\setlength{\mathindent}{0pt}

\begin{question}
	Let $f : [0,1] \to \mathbb{R}$ be defined by $f(x) = e^{-x}$.
	Consider $n + 1$ sample point $\{x_0,\dots,x_n\}$ in $[0,1]$, defined by
	\begin{align*}
		x_k & = k h \\
		h   & = \frac{1}{n}
	\end{align*}
	where $k \in \{0,\dots,n\}$.\\
	Consider the piecewise linear interpolation $l(x)$ based on this data.
	\begin{enumerate}
		\item Bound $\left| e(x) \right| = \left| f(x) - l(x) \right|$ for $x \in [0,1]$.
		\item How many samples do we need in order to guarantee an interpolation error bounded by $10^{-8}$?
	\end{enumerate}
\end{question}

\begin{solution}
	\begin{enumerate}[leftmargin=*]
		\item
			\begin{align*}
				\left| e(x) \right| & \le \frac{h^2}{8} \max\limits_{t \in [0,1]} \left| f''(x) \right| \\
                                                    & \le \frac{1}{8 n^2} \left| \left( e^{-x} \right)'' \right|        \\
                                                    & \le \frac{1}{8 n^2} \left| e^{-x} \right|                         \\
                                                    & \le \frac{1}{8 n^2}
			\end{align*}
		\item
			\begin{align*}
				\left| e(x) \right| & \le \frac{1}{8 n^2}         \\
				\therefore 10^{-8}  & \le \frac{1}{8 n^2}         \\
				\therefore n^2      & \ge \frac{10^{8}}{8}        \\
				\therefore n        & \ge \frac{10^4}{2 \sqrt{2}} \\
				\therefore n        & \ge 3536
			\end{align*}
			Therefore, we need 3536 samples to guarantee an interpolation error bounded by $10^{-8}$.
	\end{enumerate}
\end{solution}

\begin{question}
	Let $f : [0,1] \to \mathbb{R}$ be defined by $f(x) = \cos(x)$.
	Consider $n + 1$ sample point $\{x_0,\dots,x_n\}$ in $[0,1]$, defined by
	\begin{align*}
		x_k & = k h \\
		h   & = \frac{1}{n}
	\end{align*}
	where $k \in \{0,\dots,n\}$, where $n$ is even.
	Consider the piecewise quadratic interpolation $l(x)$ based on this data.
	Namely, we approximate $f(x)$ at each $x \in [x_{2 k},x_{2 k + 2}]$ by the quadratic interpolating polynomial based on the samples $\{x_{2 k},x_{2 k + 1},x_{2 k + 2}\}$.
	\begin{enumerate}
		\item Bound $\left| e(x) \right| = \left| f(x) - l(x) \right|$ for $x \in [0,\pi]$.
		\item How many samples do we need in order to guarantee an interpolation error bounded by $9 \times 10^{-5}$?
	\end{enumerate}
\end{question}

\begin{solution}
	\begin{enumerate}[leftmargin=*]
		\item
			\begin{align*}
				\left| e(x) \right| & \le \frac{\sqrt{8}}{27} \frac{\pi^3}{4 n^3} \frac{1}{3!} \max\limits_{t \in [0,\pi]} \left| f'''(x) \right| \\
                                                    & \le \frac{\sqrt{2} \pi^3}{486} \max\limits_{t \in [0,\pi]} \left| \left( \cos(x) \right)''' \right|         \\
                                                    & \le \frac{\sqrt{2} \pi^3}{486}
			\end{align*}
	\end{enumerate}
\end{solution}

\begin{question}
	Let $f$ be infinitely differentiable in $[a,b]$.
	Assume that there exists $M > 0$ such that
	\begin{align*}
		\max\limits_{a \le x \le b} \left| f^{(k)}(x) \right| & < M^k
	\end{align*}
	for any $k = \mathbb{N}$.
	Show that the error in the interpolating polynomial satisfies
	\begin{align*}
		\lim\limits_{n \to \infty} \left| e_n(x) \right| & = \lim\limits_{n \to \infty} \left| f(x) - p_n(x) \right| \\
                                                                 & = 0
	\end{align*}
\end{question}

\begin{solution}
	\begin{align*}
		\lim\limits_{n \to \infty} \left| e_n(x) \right|          & = \lim\limits_{n \to \infty} \left| f(x) - p_n(x) \right|                                                                     \\
                                                                          & = \lim\limits_{n \to \infty} \left| \frac{f^{(n + 1)}(c)}{(n + 1)!} \right| \left| \prod\limits_{i = 0}^{n} (x - x_i) \right| \\
                                                                          & \le \lim\limits_{n \to \infty} \frac{M^{n + 1}}{(n + 1)!} (b - a)^{n + 1}                                                     \\
		\therefore \lim\limits_{n \to \infty} \left| e(x) \right| & = 0
	\end{align*}
	\qed
\end{solution}

\begin{question}
	Let $f(x) = x^{\frac{99}{97}}$ in $[-h,h]$.
	Bound the linear interpolation error based on the samples $\{-h,h\}$, and show that the error is of order $\mathrm{O}\left( h^{\frac{99}{97}} \right)$.
\end{question}

\begin{solution}
	\begin{align*}
		\left| e(x) \right| & \le \left| \frac{f''(c)}{2!} \right| \left| (x + h) (x - h) \right|                                               \\
                                    & \le \frac{1}{2} \left| \frac{99}{97} \cdot \frac{2}{97} \cdot c^{-\frac{95}{97}} \right| \left| x^2 - h^2 \right| \\
                                    & \le \left| \frac{99}{(97)^2} c^{-\frac{95}{97}} \right| h^2                                                       \\
                                    & \le \left| \frac{99}{(97)^2} c^{-\frac{95}{97}} \right| h^{\frac{99}{97}}
	\end{align*}
	Therefore, $e(x)$ is of order $\mathrm{O}\left( h^{\frac{99}{97}} \right)$.
\end{solution}

\end{document}
