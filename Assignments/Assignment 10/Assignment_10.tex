\documentclass[fleqn, a4paper, 11pt, oneside, reqno]{amsart}
%\usepackage[top = 2cm, bottom = 1cm, left = 1cm, right = 1cm]{geometry}
\usepackage{exsheets, tasks}
\usepackage{amsmath, amssymb, amsthm} %standard AMS packages
\usepackage{marginnote} %marginnotes
\usepackage{gensymb} %miscellaneous symbols
\usepackage{commath} %differential symbols
\usepackage{xcolor} %colours
\usepackage{cancel} %cancelling terms
\usepackage[free-standing-units, space-before-unit]{siunitx} %formatting units
\usepackage{tikz, pgfplots} %diagrams
\usetikzlibrary{calc, hobby, patterns, intersections, decorations.markings}
\usepackage{graphicx} %inserting graphics
\usepackage{hyperref} %hyperlinks
\usepackage{datetime} %date and time
\usepackage{ulem} %underline for \emph{}
\usepackage{xfrac} %inline fractions
\usepackage{enumerate,enumitem} %numbered lists
\usepackage{float} %inserting floats
\usepackage{circuitikz}[american voltages, american currents] %circuit diagrams

\newcommand\numberthis{\addtocounter{equation}{1}\tag{\theequation}} %adds numbers to specific equations in non-numbered list of equations

\newcommand{\AxisRotator}[1][rotate=0]{
	\tikz [x=0.25cm,y=0.60cm,line width=.2ex,-stealth,#1] \draw (0,0) arc (-150:150:1 and 1);%
} %rotation symbols on axes

\theoremstyle{definition}
\newtheorem{example}{Example}
\newtheorem{definition}{Definition}

\theoremstyle{theorem}
\newtheorem{theorem}{Theorem}

\newcommand{\curl}{\mathrm{curl\,}}

\makeatletter
\@addtoreset{section}{part} %resets section numbers in new part
\makeatother

\renewcommand{\thesubsection}{(\arabic{subsection})}
\renewcommand{\thesection}{(\arabic{section})}

\DeclareMathOperator{\cond}{cond}

%section headings on left
\makeatletter
\def\specialsection{\@startsection{section}{1}%
	\z@{\linespacing\@plus\linespacing}{.5\linespacing}%
	%  {\normalfont\centering}}% DELETED
	{\normalfont}}% NEW
\def\section{\@startsection{section}{1}%
	\z@{.7\linespacing\@plus\linespacing}{.5\linespacing}%
	%  {\normalfont\scshape\centering}}% DELETED
	{\normalfont\scshape}}% NEW
\makeatother

%forces newline after subsection
\makeatletter
\def\subsection{\@startsection{subsection}{3}%
	\z@{.5\linespacing\@plus.7\linespacing}{.1\linespacing}%
	{\normalfont\itshape}}
\makeatother

%line between rows of matrix
\makeatletter
\renewcommand*\env@matrix[1][*\c@MaxMatrixCols c]{%
  \hskip -\arraycolsep
  \let\@ifnextchar\new@ifnextchar
  \array{#1}}
\makeatother

\renewcommand{\arraystretch}{1.5}

\settasks{counter-format = tsk[1].}

\SetupExSheets{solution/print = true}

%opening
\title{Numerical Analysis : Assignment 10}
\author
{
	Aakash Jog\\
	ID : 989323563
}
\date{\formatdate{29}{12}{2015}}

\begin{document}

\tikzset{->-/.style={decoration={
  markings,
  mark=at position #1 with {\arrow{>}}},postaction={decorate}}}

\maketitle
%\setlength{\mathindent}{0pt}

\begin{question}
	Let $f$ be $5$ times continuously differentiable.
	Consider the samples $f(-h)$, $f(0)$, $f(h)$.
	\begin{enumerate}
		\item
			Find the interpolating polynomial of $f$ based on the samples $0$, $h$.
			Find a formula for the interpolation error.
		\item
			Find an approximation of $f'(0)$ based on part 1.
			Find an error formula for the interpolation error.
		\item
			Find the interpolating polynomial of $f$ based on the samples $-h$, $0$, $h$.
			Find a formula for the interpolation error.
		\item
			Find an approximation of $f'(0)$ based on part 3.
			Find an error formula for your approximation.
		\item
			Find an approximation of $f''(0)$ based on part 3.
			Find an error formula for your approximation.
	\end{enumerate}
\end{question}

\begin{solution}
	\begin{enumerate}[leftmargin=*]
		\item
			\begin{align*}
				f[x_0] & = f(0) \\
				f[x_1] & = f(h)
			\end{align*}
			Therefore,
			\begin{align*}
				f[x_0,x_1] & = \frac{f(0) - f(h)}{-h}
			\end{align*}
			Therefore,
			\begin{align*}
				p_1(x) & = f[x_0] + f[x_0,x_1] (x - x_0) \\
                                       & = f(0) + \frac{f(0) - f(h)}{-h} x
			\end{align*}
			Therefore,
			\begin{align*}
				\psi(x) & = \prod (x - x_i) \\
                                        & = (x) (x - h)
			\end{align*}
			Therefore,
			\begin{align*}
				f(x) & = p_2(x) + f[x_0,x_1,x_2,x] \psi(x) \\
                                     & = p_2(x) + f[0,h,x] (x) (x - h)
			\end{align*}
			Therefore,
			\begin{align*}
				e(x) & = f[0,h,x] (x) (x - h)
			\end{align*}
		\item
			\begin{align*}
				f'(x)            & \approx {p_1}'(x) \\
				\therefore f'(0) & = \frac{f(0) - f(h)}{-h}
			\end{align*}
			Therefore,
			\begin{align*}
				e'(x) & = f[0,h,x,x] \psi(x) + f[0,h,x] \psi'(x) \\
                                      & = f[0,h,x,x] (x) (x - h) + f[0,h,x] (2 x - h)
			\end{align*}
		\item
			\begin{align*}
				f[x_0] & = f(-h) \\
				f[x_1] & = f(0)  \\
				f[x_2] & = f(h)
			\end{align*}
			Therefore,
			\begin{align*}
				f[x_0,x_1] & = \frac{f(-h) - f(0)}{-h} \\
				f[x_1,x_2] & = \frac{f(0) - f(h)}{-h}
			\end{align*}
			Therefore,
			\begin{align*}
				f[x_0,x_1,x_2] & = \frac{f(-h) - 2 f(0) + f(h)}{2 h^2}
			\end{align*}
			Therefore,
			\begin{align*}
				p_2(x) & = f[x_0] + f[x_0,x_1] (x - x_0) + f[x_0,x_1,x_2] (x - x_0) (x - x_1) \\
                                       & = f(-h) + \frac{f(-h) - f(0)}{-h} (x + h) + \frac{f(-h) - 2 f(0) + f(h)}{2 h^2} (x + h) x
			\end{align*}
			Therefore,
			\begin{align*}
				\psi(x) & = \prod (x - x_i) \\
                                        & = (x + h) (x) (x - h)
			\end{align*}
			Therefore,
			\begin{align*}
				f(x) & = p_2(x) + f[x_0,x_1,x_2,x] \psi(x) \\
                                     & = p_2(x) + f[-h,0,h,x] (x + h) (x) (x - h)
			\end{align*}
			Therefore,
			\begin{align*}
				e(x) & = f[-h,0,h,x] (x + h) (x) (x - h)
			\end{align*}
		\item
			\begin{align*}
				f'(x)            & \approx {p_2}'(x)                                                         \\
                                                 & = \frac{f(-h) - f(0)}{-h} + \frac{f(-h) - 2 f(0) + f(h)}{2 h^2} (2 x + h) \\
				\therefore f'(0) & = \frac{f(-h) - f(0)}{-h} - \frac{f(-h) - 2 f(0) + f(h)}{2 h^2} h
			\end{align*}
			Therefore,
			\begin{align*}
				e'(x) & = f[-h,0,h,x,x] \psi(x) + f[-h,0,h,x] \psi'(x) \\
                                      & = f[-h,0,h,x,x] (x + h) (x) (x - h) + f[-h,0,h,x] \left( 3 x^2 - h^2 \right)
			\end{align*}
		\item
			\begin{align*}
				f''(x)            & \approx {p_2}''(x)                      \\
                                                  & = 2 \frac{f(-h) - 2 f(0) + f(h)}{2 h^2} \\
				\therefore f''(0) & = \frac{f(-h) - 2 f(0) + f(h)}{h^2}     \\
			\end{align*}
			Therefore,
			\begin{align*}
				e''(x) & = f[-h,0,h,x,x,x] (x + h) (x) (x - h) + f[-h,0,h,x,x] (6 x)
			\end{align*}
	\end{enumerate}
\end{solution}

\begin{question}
	Let $f$ be $5$ times continuously differentiable.
	Consider the samples $f(-h)$, $f'(-h)$, $f(2 h)$.
	\begin{enumerate}
		\item
			Find the interpolating polynomial of $f$ based on these samples.
			Find a formula for the interpolation error.
		\item
			Find an approximation of $f''(0)$ based on part 1.
			Find an error formula for the interpolation error.
		\item
			Find an approximation of $f''(2 h)$ based on part 1.
			Find an error formula for the interpolation error.
	\end{enumerate}
\end{question}

\begin{solution}
	\begin{enumerate}[leftmargin=*]
		\item
			\begin{align*}
				f[x_0] & = f(-h) \\
				f[x_1] & = f(-h) \\
				f[x_2] & = f(2 h)
			\end{align*}
			Therefore,
			\begin{align*}
				f[x_0,x_1] & = f'(-h) \\
				f[x_1,x_2] & = \frac{f(2 h) - f(-h)}{3 h}
			\end{align*}
			Therefore,
			\begin{align*}
				f[x_0,x_1,x_2] & = \frac{f(2 h) - f(-h) - 3 h f'(-h)}{9 h^2}
			\end{align*}
			Therefore,
			\begin{align*}
				p_2(x) & = f[x_0] + f[x_0,x_1] (x - x_0) + f[x_0,x_1,x_2] (x - x_0) (x - x_1) \\
                                       & = f(-h) + f'(-h) (x + h) + \frac{f(2 h) - f(-h) - 3 h f'(-h)}{9 h^2} (x + h)^2
			\end{align*}
			Therefore,
			Therefore,
			\begin{align*}
				\psi(x) & = \prod (x - x_i) \\
                                        & = (x + h)^2 (x - 2 h)
			\end{align*}
			Therefore,
			\begin{align*}
				f(x) & = p_2(x) + f[x_0,x_1,x_2,x] \psi(x) \\
                                     & = p_2(x) + f[-h,-h,2 h,x] (x + h)^2 (x - 2 h)
			\end{align*}
			Therefore,
			\begin{align*}
				e(x) & = f[-h,-h,2 h,x] (x + h)^2 (x - 2 h)
			\end{align*}
		\item
			\begin{align*}
				f''(x) & \approx {p_2}''(x) \\
                                       & = \frac{f(2 h) - f(-h) - 3 h f'(-h)}{9 h^2} (2)
			\end{align*}
			Therefore,
			\begin{align*}
				f''(0) & = 2 \frac{f(2 h) - f(-h) - 3 h f'(-h)}{9 h^2}
			\end{align*}
			Therefore,
			\begin{align*}
				e''(x) & = \quad f[-h,-h,2 h,x,x,x] (x + h)^2 (x - 2 h)          \\
                                       & \quad + 2 f[-h,-h,2 h,x,x] \left( 3 x^2 - 3 h^2 \right) \\
                                       & \quad + f[-h,-h,2 h,x] (6 x)
			\end{align*}
		\item
			\begin{align*}
				f''(x) & \approx {p_2}''(x) \\
                                       & = \frac{f(2 h) - f(-h) - 3 h f'(-h)}{9 h^2} (2)
			\end{align*}
			Therefore,
			\begin{align*}
				f''(2 h) & = 2 \frac{f(2 h) - f(-h) - 3 h f'(-h)}{9 h^2}
			\end{align*}
			Therefore,
			\begin{align*}
				e''(x) & = \quad f[-h,-h,2 h,x,x,x] (x + h)^2 (x - 2 h)          \\
                                       & \quad + 2 f[-h,-h,2 h,x,x] \left( 3 x^2 - 3 h^2 \right) \\
                                       & \quad + f[-h,-h,2 h,x] (6 x)
			\end{align*}
	\end{enumerate}
\end{solution}

\end{document}
