\documentclass[fleqn, a4paper, 11pt, oneside]{amsart}
%\usepackage[top = 2cm, bottom = 1cm, left = 1cm, right = 1cm]{geometry}
\usepackage{exsheets, tasks}
\usepackage{amsmath, amssymb, amsthm} %standard AMS packages
\usepackage{marginnote} %marginnotes
\usepackage{gensymb} %miscellaneous symbols
\usepackage{commath} %differential symbols
\usepackage{xcolor} %colours
\usepackage{cancel} %cancelling terms
\usepackage[free-standing-units, space-before-unit]{siunitx} %formatting units
\usepackage{tikz, pgfplots} %diagrams
\usetikzlibrary{calc, hobby, patterns, intersections, decorations.markings}
\usepackage{graphicx} %inserting graphics
\usepackage{hyperref} %hyperlinks
\usepackage{datetime} %date and time
\usepackage{ulem} %underline for \emph{}
\usepackage{xfrac} %inline fractions
\usepackage{enumerate,enumitem} %numbered lists
\usepackage{float} %inserting floats
\usepackage{circuitikz}[american voltages, american currents] %circuit diagrams

\newcommand\numberthis{\addtocounter{equation}{1}\tag{\theequation}} %adds numbers to specific equations in non-numbered list of equations

\newcommand{\AxisRotator}[1][rotate=0]{
	\tikz [x=0.25cm,y=0.60cm,line width=.2ex,-stealth,#1] \draw (0,0) arc (-150:150:1 and 1);%
} %rotation symbols on axes

\theoremstyle{definition}
\newtheorem{example}{Example}
\newtheorem{definition}{Definition}

\theoremstyle{theorem}
\newtheorem{theorem}{Theorem}

\newcommand{\curl}{\mathrm{curl\,}}

\makeatletter
\@addtoreset{section}{part} %resets section numbers in new part
\makeatother

\renewcommand{\thesubsection}{(\arabic{subsection})}
\renewcommand{\thesection}{(\arabic{section})}

%section headings on left
\makeatletter
\def\specialsection{\@startsection{section}{1}%
	\z@{\linespacing\@plus\linespacing}{.5\linespacing}%
	%  {\normalfont\centering}}% DELETED
	{\normalfont}}% NEW
\def\section{\@startsection{section}{1}%
	\z@{.7\linespacing\@plus\linespacing}{.5\linespacing}%
	%  {\normalfont\scshape\centering}}% DELETED
	{\normalfont\scshape}}% NEW
\makeatother

%forces newline after subsection
\makeatletter
\def\subsection{\@startsection{subsection}{3}%
	\z@{.5\linespacing\@plus.7\linespacing}{.1\linespacing}%
	{\normalfont\itshape}}
\makeatother

\settasks{counter-format = tsk[1].}

\SetupExSheets{solution/print = true}

%opening
\title{Numerical Analysis : Assignment 3}
\author
{
	Aakash Jog\\
	ID : 989323563
}
\date{\formatdate{3}{11}{2015}}

\begin{document}

\tikzset{->-/.style={decoration={
  markings,
  mark=at position #1 with {\arrow{>}}},postaction={decorate}}}

\maketitle
%\setlength{\mathindent}{0pt}

\begin{question}
	Show that $f[x_0,\dots,x_n]$ does not depend on the order of the points $\{x_0,\dots,x_n\}$.
\end{question}

\begin{solution}
	If $n = 1$,
	\begin{align*}
		f[x_0,x_1]            & = \frac{f(x_0) - f(x_1)}{x_0 - x_1} \\
		f[x_1,x_0]            & = \frac{f(x_1) - f(x_0)}{x_1 - x_0} \\
		\therefore f[x_0,x_1] & = f[x_1,x_0]
	\end{align*}
	If possible let
	\begin{align*}
		f[x_0,\dots,x_k,x_{k + 1},\dots,x_n] & = f[x_0,\dots,x_{k + 1},x_k,\dots,x_n]
	\end{align*}
	Therefore,
	\begin{align*}
		f[x_0,\dots,x_{n + 1}] & = \frac{f[x_1,\dots,x_k,x_{k + 1},\dots,x_{n + 1}] - f[x_0,\dots,x_k,x_{k + 1},\dots,x_n]}{x_{n + 1} - x_0} \\
                                       & = \frac{f[x_1,\dots,x_{k + 1},x_k,\dots,x_{n + 1}] - f[x_0,\dots,x_{k + 1},x_k,\dots,x_n]}{x_{n + 1} - x_0} \\
                                       & = f[x_0,\dots,x_{k + 1},x_k,\dots,x_{n + 1}]
	\end{align*}
	Therefore, by induction, $f[x_0,\dots,x_n]$ does no depend on the order of the points $\{x_0,\dots,x_n\}$.
	\qed
\end{solution}

\addtocounter{question}{1}

\begin{question}
	Let $f : [0,1] \to \mathbb{R}$ be defined as
	\begin{align*}
		f(x) & = e^{-x}
	\end{align*}
	Consider the $n + 1$ sample points $\{x_0,\dots,x_n\}$ in $[0,1]$, and the interpolating polynomial $p_n(x)$ of $f(x)$ at these points.
	\begin{enumerate}
		\item
			Show that $\forall x \in [0,1]$,
			\begin{align*}
				\left| (x - x_0) \dots (x - x_n) \right| & \le 1
			\end{align*}
		\item
			Prove
			\begin{align*}
				\left| e_n(x) \right| & = \left| f(x) - p_n(x) \right| \\
                                                      & \le \frac{1}{n!}
			\end{align*}
		\item
			How many sample points do we need if we require an approximation of $f$ with error less than $10^{-3}$?
	\end{enumerate}
\end{question}

\begin{solution}
	\begin{enumerate}[leftmargin = *]
		\item
			As $x \in [0,1]$, and all $x_j \in [0,1]$,
			\begin{align*}
				|x - x_j| & \le 1
			\end{align*}
			for every $j$.\\
			Therefore,
			\begin{align*}
				\prod (x - x_i) & \le 1
			\end{align*}
			\qed
		\item
			\begin{align*}
				\left| e_n(x) \right| & = \left| f(x) - p_n(x) \right|                                                        \\
                                                      & \le \left| \frac{f^{(n + 1)}(c)}{(n + 1)!} \prod\limits_{j = 0}^{n} (x - x_j) \right| \\
                                                      & \le \left| \frac{(-1)^{n}f(c)}{n!} \prod\limits_{j = 0}^{n} (x - x_j) \right|         \\
                                                      & \le \left| \frac{e^{-c}}{n!} \right|
			\end{align*}
			Therefore, as $c \in [0,1]$,
			\begin{align*}
				\left| e_n(x) \right| & \le \frac{1}{n!}
			\end{align*}
		\item
			\begin{align*}
				\left| e_n(x) \right|   & \le \frac{1}{n!} \\
				\therefore \frac{1}{n!} & \le 10^{-3}      \\
				\therefore n!           & \ge 10^3
			\end{align*}
			Therefore,
			\begin{align*}
				n & \ge 7
			\end{align*}
			Therefore, we need at least 7 sample points if we require an approximation of $f$ with error less than $10^{-3}$.
	\end{enumerate}
\end{solution}

\end{document}
