\documentclass[fleqn, a4paper, 12pt, twoside, titlepage]{article}
\usepackage{exsheets} %question and solution environments
\usepackage{amsmath, amssymb, amsthm} %standard AMS packages
\usepackage{esint} %integral signs
\usepackage{marginnote} %marginnotes
\usepackage{gensymb} %miscellaneous symbols
\usepackage{commath} %differential symbols
\usepackage{xcolor} %colours
\usepackage{cancel} %cancelling terms
\usepackage{siunitx} %formatting units
\usepackage{tikz, pgfplots} %diagrams
	\usetikzlibrary{calc, hobby, patterns, intersections, angles, quotes, spy}
\usepackage{graphicx} %inserting graphics
\usepackage{epstopdf} %converting and inserting eps graphics
\usepackage{hyperref} %hyperlinks
\usepackage{datetime} %date and time
\usepackage{ulem} %underline for \emph{}
\usepackage{xfrac, lmodern} %inline fractions
\usepackage{enumerate, enumitem} %numbered lists
\usepackage{float} %inserting floats
\usepackage[american voltages]{circuitikz} %circuit diagrams
\usepackage{pdflscape} %pages in landscape orientation
\usepackage{setspace} %double spacing
\usepackage{microtype} %micro-typography
\usepackage{listings} %formatting code
	\lstset{language=Matlab}
	\lstdefinestyle{standardMatlab}
	{
		belowcaptionskip=1\baselineskip,
		breaklines=true,
		frame=L,
		xleftmargin=\parindent,
		language=C,
		showstringspaces=false,
		basicstyle=\footnotesize\ttfamily,
		keywordstyle=\bfseries\color{green!40!black},
		commentstyle=\itshape\color{purple!40!black},
		identifierstyle=\color{blue},
		stringstyle=\color{orange},
	}
\usepackage{algpseudocode} %algorithms
\usepackage{algorithm} %algorithms
\usepackage{todonotes}

\newcommand\numberthis{\addtocounter{equation}{1}\tag{\theequation}} %adds numbers to specific equations in non-numbered list of equations

\theoremstyle{definition}
\newtheorem{example}{Example}
\newtheorem{definition}{Definition}

\theoremstyle{theorem}
\newtheorem{theorem}{Theorem}
\newtheorem{law}{Law}

\newcommand{\curl}{\mathrm{curl\,}}

\newcommand{\divergence}{\mathrm{div\,}}

\renewcommand{\algorithmicrequire}{\textbf{Input:}}
\renewcommand{\algorithmicensure}{\textbf{Output:}}

\DeclareMathOperator{\cond}{cond}

\makeatletter
\@addtoreset{section}{part} %resets section numbers in new part
\makeatother

\newcommand\blfootnote[1]{%
	\begingroup
	\renewcommand\thefootnote{}\footnote{#1}%
	\addtocounter{footnote}{-1}%
	\endgroup
}

\renewcommand{\tilde}{\widetilde}

\renewcommand{\marginfont}{\scriptsize \color{blue}}

\SetupExSheets{solution/print = true} %prints all solutions by default

%opening
\title{Numerical Analysis : Review Session}
\author{Aakash Jog}
\date{2015-16}

\begin{document}

\maketitle
%\setlength{\mathindent}{0pt}

\blfootnote
{	
	\begin{figure}[H]
		\includegraphics[height = 12pt]{cc.eps}
		\includegraphics[height = 12pt]{by.eps}
		\includegraphics[height = 12pt]{nc.eps}
		\includegraphics[height = 12pt]{sa.eps}
	\end{figure}
	This work is licensed under the Creative Commons Attribution-NonCommercial-ShareAlike 4.0 International License. To view a copy of this license, visit \url{http://creativecommons.org/licenses/by-nc-sa/4.0/}.
} %CC-BY-NC-SA license

\tableofcontents

\newpage

\begin{question}
	Let $f : [0,1] \to \mathbb{R}$.
	Consider the partition $0 < \frac{1}{N} < \dots < \frac{N - 1}{N} < 1$, for $N \in \mathbb{N}$.
	Consider the space $\mathit{PC}$, i.e. the space of piecewise constant functions on this partition.
	Therefore, $g \in \mathit{PC}$, if it is constant on each $\left[ \frac{n}{N},\frac{n + 1}{N} \right]$, where $n = 0,\dots,N - 1$.
	Find a formula for $g \in \mathit{PC}$, closest to $f$, in the sense that $\displaystyle \int\limits_{0}^{1} \left| f(x) - g(x) \right|^2 \dif x$ is minimal.
\end{question}

\begin{solution}
	Let
	\begin{align*}
		\varphi_n(x) &=
			\begin{cases}
				1 &;\quad \frac{n}{N} \le x \le \frac{n + 1}{N}\\
				0 &;\quad \text{otherwise}
			\end{cases}
	\end{align*}
	Therefore, the set of all $\varphi_n(x)$ form a basis of the space.
	Therefore, let
	\begin{align*}
		g(x) &= \sum\limits_{n = 0}^{N - 1} c_n \varphi_n
	\end{align*}
	where $c_n \in \mathbb{R}$.
	Therefore, the Gram matrix is
	\begin{align*}
			\begin{pmatrix}
				\langle \varphi_0,\varphi_0 \rangle & \dots & \langle \varphi_0,\varphi_{N - 1} \rangle\\
				\vdots & &\\
				\langle \varphi_{N - 1},\varphi_0 \rangle & \dots & \langle \varphi_{N - 1},\varphi_{N - 1} \rangle\\
			\end{pmatrix}
			\begin{pmatrix}
				c_0\\
				\vdots\\
				c_{N - 1}\\
			\end{pmatrix}
		&=
			\begin{pmatrix}
				\langle f,\varphi_0 \rangle\\
				\vdots\\
				\langle f,\varphi_{N - 1} \rangle\\
			\end{pmatrix}
	\end{align*}
	Therefore, as
	\begin{align*}
		\int\limits_{0}^{1} \left| f(x) - g(x) \right|^2 \dif x &= \|f - g\|^2
	\end{align*}
	Therefore,
	\begin{align*}
		\langle \alpha,\beta \rangle &= \int\limits_{0}^{1} \alpha(x) \overline{\beta(x)} \dif x
	\end{align*}
	Therefore,
	\begin{align*}
		\langle \varphi_n,\varphi_m \rangle &= \int\limits_{0}^{1} \varphi_n(x) \varphi_m(x) \dif x
	\end{align*}
	If $n \neq m$,
	\begin{align*}
		\int\limits_{0}^{1} \varphi_n(x) \varphi_m(x) \dif x &= 0
	\end{align*}
	If $n = m$,
	\begin{align*}
		\int\limits_{0}^{1} \varphi_n(x) \varphi_m(x) \dif x &= \int\limits_{0}^{1} \varphi_n(x) \varphi_n(x) \dif x\\
		&= \int\limits_{\frac{n}{N}}^{\frac{n + 1}{N}} 1 \dif x\\
		&= \frac{1}{N}
	\end{align*}
	Therefore,
	\begin{align*}
		\langle \varphi_n,\varphi_m \rangle &=
			\begin{cases}
				0 &;\quad n \neq m\\
				\frac{1}{N} &;\quad n = m\\
			\end{cases}
	\end{align*}
	Therefore,
	Similarly,
	\begin{align*}
		\langle f,\varphi_n \rangle &= \int\limits_{\frac{n}{N}}^{\frac{n + 1}{N}} f(x) \dif x
	\end{align*}
	Therefore,
	\begin{align*}
			\begin{pmatrix}
				\frac{1}{N} & \dots & 0\\
				\vdots & &\\
				0 & \dots & \frac{1}{N}\\
			\end{pmatrix}
			\begin{pmatrix}
				c_0\\
				\vdots\\
				c_{N - 1}\\
			\end{pmatrix}
		&=
			\begin{pmatrix}
				\int\limits_{0}^{\frac{1}{N}} f(x) \dif x\\
				\vdots\\
				\int\limits_{\frac{N - 1}{N}}^{1} f(x) \dif x\\
			\end{pmatrix}\\
		\therefore
			\begin{pmatrix}
				c_0\\
				\vdots\\
				c_{N - 1}\\
			\end{pmatrix}
		&=
			\begin{pmatrix}
				N & \dots & 0\\
				\vdots & &\\
				0 & \dots & N\\
			\end{pmatrix}
			\begin{pmatrix}
				\int\limits_{0}^{\frac{1}{N}} f(x) \dif x\\
				\vdots\\
				\int\limits_{\frac{N - 1}{N}}^{1} f(x) \dif x\\
			\end{pmatrix}
	\end{align*}
	Therefore,
	\begin{align*}
		g(x) &= \sum\limits_{n = 0}^{N - 1} N \left( \int\limits_{\frac{n}{N}}^{\frac{n + 1}{N}} f(t) \dif t \right) \varphi_n(x)
	\end{align*}
\end{solution}

\begin{question}
	Consider the following system.
	\begin{align*}
			\begin{pmatrix}
				1 & \frac{1}{4}\\
				p & \frac{1}{2}\\
			\end{pmatrix}
			\begin{pmatrix}
				x_1\\
				x_2\\
			\end{pmatrix}
		&=
			\begin{pmatrix}
				1\\
				1\\
			\end{pmatrix}
	\end{align*}
	\begin{enumerate}
		\item Write the Gauss-Seidel iteration scheme.
		\item Find all values of $p$ for which the method converges, for any $x^{(0)}$.
		\item Given that $p = \frac{1}{4}$, how many iterations are required in order to reduce the initial error $\left\| x - x^{(0)} \right\|$ by six orders of magnitude, i.e. $10^{-6}$.
	\end{enumerate}
\end{question}

\begin{solution}
	\begin{enumerate}[leftmargin=*]
		\item
			For Gauss-Seidel,
			\begin{align*}
				x^{(n + 1)} &= B x^{(n)} + d
			\end{align*}
			where
			\begin{align*}
				B &= -(L + D)^{-1} U\\
				d &= (L + D)^{-1} b
			\end{align*}
			Therefore,
			\begin{align*}
				L + D &=
					\begin{pmatrix}
						1 & 0\\
						p & \frac{1}{2}\\
					\end{pmatrix}\\
				\therefore (L + D)^{-1} &=
					\begin{pmatrix}
						1 & 0\\
						p & \frac{1}{2}\\
					\end{pmatrix}^{-1}
				&=
					2
					\begin{pmatrix}
						\frac{1}{2} & 0\\
						-p & 1\\
					\end{pmatrix}
			\end{align*}
			Therefore,
			\begin{align*}
				B &=
					\begin{pmatrix}
						-1 & 0\\
						2 p & -2\\
					\end{pmatrix}
					\begin{pmatrix}
						0 & \frac{1}{4}\\
						0 & 0\\
					\end{pmatrix}\\
				&=
					\begin{pmatrix}
						0 & -\frac{1}{4}\\
						0 & \frac{p}{2}\\
					\end{pmatrix}\\
				d &=
					\begin{pmatrix}
						1 & 0\\
						-2 p & 2\\
					\end{pmatrix}
					\begin{pmatrix}
						1\\
						1\\
					\end{pmatrix}
				&=
					\begin{pmatrix}
						1\\
						2 - 2 p\\
					\end{pmatrix}
			\end{align*}
			Therefore,
			\begin{align*}
				x^{(n + 1)} &=
					\begin{pmatrix}
						0 & -\frac{1}{4}\\
						0 & \frac{p}{2}\\
					\end{pmatrix}
					x^{(n)}
					+
					\begin{pmatrix}
						1\\
						2 - 2 p\\
					\end{pmatrix}
			\end{align*}
		\item
			For the method to converge,
			\begin{align*}
				\rho(B) &< 1
			\end{align*}
			where $\rho$ is the spectral radius of $B$.\\
			Therefore,
			\begin{align*}
				\det(\lambda I - B) &=
					\det
					\begin{pmatrix}
						\lambda & \frac{1}{4}\\
						0 & \lambda - \frac{p}{2}
					\end{pmatrix}\\
				&= \lambda \left( \lambda - \frac{p}{2} \right)
			\end{align*}
			Therefore,
			\begin{align*}
				\lambda_1 &= 0\\
				\lambda_2 &= \frac{p}{2}
			\end{align*}
			Therefore, as
			\begin{align*}
				\rho(B) &< 1
			\end{align*}
			Therefore,
			\begin{align*}
				|\lambda_2| &< 1\\
				\therefore |p| &< 2
			\end{align*}
		\item
			\begin{align*}
				e^{(n)} &= x - x^{(n)}\\
				&= B^n e^{(0)}
			\end{align*}
			Therefore,
			\begin{align*}
				\left\| e^{(n)} \right\| &= \left\| B^n e^{(0)} \right\|\\
				&\le \left\| B^n \right\| \left\| e^{(0)} \right\|\\
				&\le \|B\|^n \left\|e^{(0)} \right\|
			\end{align*}
			The infinity norm of $B$ is
			\begin{align*}
				\|B\|_{\infty} &= \frac{1}{4}
			\end{align*}
			Therefore,
			\begin{align*}
				\left\| e^{(n)} \right\| &\le 2^{-2 n} \left\| e^{(0)} \right\|
			\end{align*}
			Therefore, for the error to be less than $10^{-6}$,
			\begin{align*}
				2^{-2 n} &< 10^{-6}
			\end{align*}
			Therefore,
			\begin{align*}
				n &= 10
			\end{align*}
	\end{enumerate}
\end{solution}

\begin{question}
	Given
	\begin{align*}
		\left( x_i,f(x_i) \right) &= \left( (1,3),(2,16),(3,11) \right)\\
		\left| f^{(3)}(x) \right| &< 5\\
		\left| f^{(4)}(x) \right| &< 4\\
		\left| f^{(5)}(x) \right| &< 3
	\end{align*}
	\begin{enumerate}
		\item
			What is the degree of the interpolation polynomial?
		\item
			Find an upper bound for the error.
		\item
			What is the degree of the interpolation polynomial after adding the sample $(4,27)$?
		\item
			Let $l_i(x)$ be the Lagrange polynomials based on the four points above.
			Find the degree of
			\begin{align*}
				Q(x) &= l_1(x) + 2 l_2(x) + 3 l_3(x) + 4 l_4(x)
			\end{align*}
	\end{enumerate}
\end{question}

\begin{solution}
	\begin{enumerate}[leftmargin=*]
		\item
			\begin{align*}
				f[x_0] &= 3\\
				f[x_1] &= 16\\
				f[x_2] &= 11
			\end{align*}
			Therefore,
			\begin{align*}
				f[x_0,x_1] &= \frac{3 - 16}{1 - 2}\\
				&= 13\\
				f[x_1,x_2] &= \frac{16 - 11}{2 - 3}\\
				&= -5
			\end{align*}
			Therefore,
			\begin{align*}
				f[x_0,x_1,x_2] &= \frac{13 - (-5)}{1 - 3}\\
				&= -9
			\end{align*}
			Therefore,
			\begin{align*}
				p(x) &= f[x_0] + f[x_0,x_1] (x - x_0) + f[x_0,x_1,x_2] (x - x_0) (x - x_1)\\
				&= 3 + 13 (x - 1) - 9 (x - 1) (x - 2)
			\end{align*}
			Therefore, the degree of the interpolating polynomial is $2$.
		\item
			\begin{align*}
				e &= f[x_0,x_1,x_2,x] (x - x_1) (x - x_2) (x - x_3)\\
				&= f[1,2,3,x] (x - 1) (x - 2) (x - 3)\\
				&\le \frac{f^{(3)}(\xi)}{3!} (x - 1)(x - 2) (x - 3)
			\end{align*}
			Therefore, according to the given conditions,
			\begin{align*}
				|e| &\le \frac{5}{6} \left| (x - 1) (x - 2) (x - 3) \right|
			\end{align*}
		\item
			\begin{align*}
				f[x_0] &= 3\\
				f[x_1] &= 16\\
				f[x_2] &= 11\
				f[x_3] &= 27
			\end{align*}
			Therefore,
			\begin{align*}
				f[x_0,x_1] &= \frac{3 - 16}{1 - 2}\\
				&= 13\\
				f[x_1,x_2] &= \frac{16 - 11}{2 - 3}\\
				&= -5\\
				f[x_2,x_3] &= \frac{11 - 27}{3 - 4}\\
				&= 16
			\end{align*}
			Therefore,
			\begin{align*}
				f[x_0,x_1,x_2] &= \frac{13 - (-5)}{1 - 3}\\
				&= -9\\
				f[x_1,x_2,x_3] &= \frac{-5 - 16}{2 - 4}\\
				&= 10.5
			\end{align*}
			Therefore,
			\begin{align*}
				f[x_0,x_1,x_2,x_3] &= \frac{-9 - 10.5}{1 - 4}\\
				&\neq 0
			\end{align*}
			Therefore, the degree of the interpolation polynomial is $3$.
		\item
			\begin{align*}
				l_1(x) &= \frac{(x - 2) (x - 3) (x - 4)}{(1 - 2) (1 - 3) (1 - 4)}\\
				l_2(x) &= \frac{(x - 1) (x - 3) (x - 4)}{(2 - 1) (2 - 3) (2 - 4)}\\
				l_3(x) &= \frac{(x - 1) (x - 2) (x - 4)}{(3 - 1) (3 - 2) (3 - 4)}\\
				l_4(x) &= \frac{(x - 1) (x - 2) (x - 3)}{(4 - 1) (4 - 2) (4 - 3)}
			\end{align*}
			Therefore, solving, the degree is $1$.
	\end{enumerate}
\end{solution}

\begin{question}
	Consider
	\begin{align*}
		x_{n + 1} &= x_n + A \frac{{x_n}^2 - 2}{x_n} + B \frac{{x_n}^2 - 2}{{x_n}^3}
	\end{align*}
	\begin{enumerate}
		\item What is the fixed point?
		\item Find $A$ and $B$, such that the rate of convergence is maximal.
	\end{enumerate}
\end{question}

\begin{solution}
	\begin{enumerate}[leftmargin=*]
		\item
			For the fixed point,
			\begin{align*}
				x &= g(x)
			\end{align*}
			Therefore,
			\begin{align*}
				x &= x + A \frac{x - 2}{x} + B \frac{x^2 - 2}{x^3}
			\end{align*}
			Therefore,
			\begin{align*}
				A \frac{x^2 - 2}{x} + B \frac{x^2 - 2}{x^3} &= 0
			\end{align*}
			Therefore, solving,
			\begin{align*}
				x &= \pm \sqrt{2}
			\end{align*}
			or
			\begin{align*}
				x &= \pm \sqrt{-\frac{A}{B}}
			\end{align*}
		\item
			\begin{align*}
				g(x) &= A \frac{x^2 - 2}{x} + B \frac{x^2 - 2}{x^3} + x\\
				\therefore g'(x) &= 1 + A \frac{2 x^2 - \left( x^2 - 2 \right)}{x^2} + B \frac{2 x^4 - \left( x^2 - 2 \right) 3 x^2}{x^6}
			\end{align*}
			Therefore, if $g'(x) = 0$, for $x = \pm \sqrt{2}$, solving,
			\begin{align*}
				1 + 2 A + B &= 0
			\end{align*}
			Therefore, if $g'(x) = 0$, for $x = \pm \sqrt{2}$, solving,
			\begin{align*}
				2 A + 5 B &= 0
			\end{align*}
			Therefore, solving,
			\begin{align*}
				A &= -\frac{5}{8}\\
				B &= \frac{1}{4}
			\end{align*}
	\end{enumerate}
\end{solution}

\end{document}
